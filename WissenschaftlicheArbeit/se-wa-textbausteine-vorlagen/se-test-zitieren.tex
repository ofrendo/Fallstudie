%  Testdatei f\"ur die Erzeugung von Literaturreferenzen, die den Regeln von Rene Theisen 
%  (Wissenschaftliches Arbeiten, 2009) folgen
%
%
%
\section{Kommandos f\"ur die Erzeugung von Literaturverweisen}

\subsection{Kurzzitierweise mit der Angabe eines Kurztitels}

Das Kommando \verb+\seCite{par1}{par2}{par3}+ erzeugt einen Literaturverweis im Text. 

\begin{seToplist}{\texttt{par1}:}
\item[\texttt{par1}:] Der erste Parameter  definiert einen optionalen Text, der vor dem eigentlichen Literaturverweis ausgegeben 
                               wird, typischerweise Vgl. oder vgl.
\item[\texttt{par2}:] Der zweite Parameter  wird verwendet, um (z.\,B.) zus\"atzliche Seitenangaben f\"ur den Literaturverweis 
                              vorzunehmen.
\item[\texttt{par2}:] Der dritte Parameter ist der entsprechende Schl\"ussel in der .bib-Datei, in der die Literaturquellen 
                              beschrieben sind (vgl. \texttt{wa.bib}).                                                                                       
\end{seToplist}

Als Beispiel f\"ur die Verwendung des \verb+\seCite+-Befehls dient folgendes Zitat: \glqq{}Die \textbf{Funktion} eines 
Anhangs einer wissenschaftlichen Arbeit wird sehr h\"aufig \textbf{missdeutet}, der Anhang selbst nicht selten \textbf{mi{\ss}braucht}.\grqq{} 
(\seCite{vgl.}{S. 170}{The:WA}).

Bei der von Theisen vorgeschlagenen Zitierweise erfolgt die Angabe der Literaturverweise in der Regel innerhalb einer Fu{\ss}note. 
Hierf\"ur kann das Kommando \verb+\seFootcite+ verwendet werden, das dieselben Parameter wie \verb+\seCite+ besitzt. 

Als Beispiel f\"ur ein indirektes Zitat l\"asst sich die Aussage von Theisen anf\"uhren, dass Hauptinhalte eines (berechtigten) Anhangs erg\"anzende 
Materialien und Dokumente sind, die weitere themenbezogene Informationen liefern k\"onnen.\seFootcite{Vgl.}{S. 171}{The:WA}

Weder das \verb+\seFootcite+- noch das \verb+\footnote+-Kommande k\"onnen bei Gleitobjekten (Verwendung der \verb+figure+-, \verb+table+- oder 
\verb+programm+-Umgebung) verwendet werden. Ein kleiner Workaround, um \LaTeX{} doch dazu zu bringen, Fu{\ss}noten bei Gleitobjekten 
zu akzeptieren, ist in \vref{gleitobjekte} zu finden.

\subsection{Harvard-Zitierweise}

Um die Harvard-Zitierweise anzuwenden, muss in der Konfigurationsdatei \newline
\hspace*{\fill}\verb+wa-konfiguration+\hspace*{\fill}\newline
die Zeile \newline
\hspace*{\fill}\verb+\usepackage{\seWaPathSty/se-jb-jurabib-theisen}+\hspace*{\fill}\newline
durch\newline
\hspace*{\fill}\verb+\usepackage{\seWaPathSty/se-jb-jurabib-harvard}+\hspace*{\fill}\newline
ausgetauscht werden. Das Literaturverzeichnis wird dann ohne die Angabe von 
Kurztiteln ausgegeben und die Autorennamen werden im Text nicht mehr kursiv 
dargestellt.

F\"ur das Zitieren im Text werden dann die Kommandos \verb+\citep+ bzw. \verb+\citealt+ 
verwendet.

\begin{seList}
\item \verb+\citep{Bri:WA}+ $\rightarrow$ \citep{Bri:WA}.
\item Optionale Angabe einer Seitenzahl: \verb+\citep[S.\,45]{Bri:WA}+ \newline$\rightarrow$ \citep[S.\,45]{Bri:WA}.
\item Optionale Angabe von vgl.: \verb+\citep[vgl.][]{Bri:WA}+ $\rightarrow$ \citep[vgl.][]{Bri:WA}.
\item Optionale Angabe von Seitenzahl und vgl.: \verb+\citep[vgl.][S.\,45]{Bri:WA}+\newline$\rightarrow$ \citep[vgl.][S.\,45]{Bri:WA}.
\item Angabe einer Liste von Referenzen: \verb+\citep[vgl.][]{Bri:WA,RP:WA}+\newline $\rightarrow$ \citep[vgl.][]{Bri:WA,RP:WA}.
\item Wenn man keine Klammern haben m\"ochte: \newline\verb+\citealt[vgl.][S.\ 44]{Bri:WA}+ $\rightarrow$ 
        \citealt[vgl.][S.\ 44]{Bri:WA}.
\item Damit kann man dann Verweise selbst zusammenbauen:
         \newline\verb+(\citealt[vgl.][S.\,48]{Bri:WA} und \citealt[S.\,96]{RP:WA})+      
         \newline$\rightarrow$ (\citealt[vgl.][S.\,48]{Bri:WA} und \citealt[S.\,96]{RP:WA}).
\end{seList}


\subsection{Verwendung von URLs}

URLs k\"onnen in der \texttt{bib}-Datei mit \texttt{@WWW} definiert werden. Das Feld \texttt{author} ist zwar optional, sollte aber immer angegeben 
werden, da andernfalls im Literaturverzeichnis der Kurztitel nicht ausgegeben wird. Wenn kein Autor bekannt ist, wird die Abk\"urzung o.\ V. verwendet.
Beim Eintrag in der \texttt{bib}-Datei ist zu beachten, dass diese Abk\"urzung zus\"atzlich eingeklammert werden muss, d.\,h. sie ist in der 
Form \texttt{author = \{\{o.~V.\}\}} anzugeben. 

Das Layout der URL-Angabe im Literaturverzeichnis kann \"uber vier Parameter beeinflusst werden.  
In der Datei \texttt{wa-konfiguration-deutsch.tex} k\"onnen Redefinitionen vorgenommen werden.

\begin{seList}
\item \verb+\biburlprefix+ \newline Text, der vor dem eigentlichen URL-Eintrag ausgegeben wird \newline Standardwert: \glqq{}\jblangle{}URL: \grqq{}
\item \verb+\biburlsuffix+ \newline Text, der hinter dem eigentlichen URL-Eintrag ausgegeben wird \newline Standardwert: \glqq{}\jbrangle{}\grqq{}
\item \verb+\bibbudcsep+ \newline Text zwischen dem eigentlichen URL-Eintrag und der Datumsangabe f\"ur den letzten Zugriff auf die URL
                                         \newline Standardwert: \glqq{} -- \grqq{}
\item \verb+\urldatecomment+ \newline Text, der vor der Datumsangabe f\"ur den letzten Zugriff ausgegeben wird
                                          \newline Standardwert: \glqq{}Zugriff am\grqq{}
\end{seList}

Und hier kommen noch zwei Beispiele f\"ur die Angabe von Literaturreferenzen, deren Quelle eine URL ist:
Das Paket \texttt{jurabib.sty} wurde von Jens Berger entwickelt.\seFootcite{Vgl.}{}{Ber:Hoj} 
\glqq{}Google will seine Suche auch in Deutschland um eine Datenbank mit abgesicherten Fakten, Biografien und Bildern erweitern, 
den Knowledge Graph.\grqq{}\seFootcite{}{}{GKN}
