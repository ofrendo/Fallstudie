\chapter{Die Erstellung englischsprachiger wissenschaftlicher Arbeiten}

\"Uber das Kommando \verb+\seWaSprache+, das am Anfang einer Vorlagendatei stehen muss,
kann die zu verwendende Sprache, Deutsch oder Englisch, festgelegt werden:

\begin{seList}
\item \verb+\newcommand{\seWaSprache}{deutsch}+  bzw.
\item \verb+\newcommand{\seWaSprache}{englisch}+
\end{seList}

Hiermit werden automatisch alle relevanten Bezeichnungen, z.\,B. auf dem Titelblatt oder im Literaturverzeichnis, aber auch Standardtexte,  
wie die ehrenw\"ortliche Erkl\"arung oder ein Sperrvermerk, in der gew\"ahlten Sprache ausgegeben.

Bei der Verwendung von Querverweisen ist zu beachten, dass im Englischen die Gro{\ss}- und Kleinschreibung zu unterscheiden ist. 
Will man beispielsweise einen Querverweis auf eine Tabelle erzeugen, der direkt am Satzanfang steht, dann ist statt des 
Kommandos \verb+\vref+ das Kommando \verb+\Vref+ zu verwenden. Entsprechend gibt es f\"ur das Kommando \verb+\vref*+ 
die Variante \verb+\Vref*+, die den ersten Buchstaben der Ausgabe (automatisch) gro{\ss} schreibt. N\"aheres zur Verwendung des 
\verb+\vref*+-Kommandos findet man in \vref{vrefstern}.

Alle \"Anderungen, die bei der deutschen Version einer Arbeit in der Datei \newline
\hspace*{\fill}\verb+wa-konfiguration-deutsch.tex+\hspace*{\fill}\newline
vorgenommen werden, m\"ussen bei einer englischsprachigen Arbeit in der Datei \newline 
\hspace*{\fill}\verb+wa-konfiguration-englisch.tex+\hspace*{\fill}\newline
erfolgen.