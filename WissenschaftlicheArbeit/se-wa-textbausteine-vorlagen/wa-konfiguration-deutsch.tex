% wa-konfiguration-deutsch
%
% 2012-12-13
% 
% Diese Datei wird f\"ur die Sprachoption deutsch verwendet, d. h.  
% \newcommand{\seWaSprache}{deutsch}
%
%
% In dieser Datei k\"onnen Neudefinitionen vorgenommen werden f\"ur:
% -- Verzeichnisse
% -- Unter-/\"Uberschriften von Abbildungen, Tabellen und Listings
% -- Querverweise innerhalb des Textes

%
%  Konfiguration der verschiedenen Verzeichnisse
%
%  abstandEintrag: Wert wird mit \baselineskip multipliziert
%

%
%  Abbildungsverzeichnis
%
\seKonfigurationAbb[
%verzeichnisname=Abbildungsverzeichnis,
labeltextLinks=, % kein Text links;
%labeltextRechts=:,
labelbreite=1cm,
%labeleinzug=1cm,
%abstandEintrag=1,
%newpage=ja,
%pnumwidth=20mm,
%dotsep=1000,
%tocrmarg=4.5cm,
%abstandVerzeichnis=-1mm
]

%
% LIstingverzeichnis
%
\seKonfigurationPrg[
%verzeichnisname=Listing-Verzeichnis,
labeltextLinks=,
%labeltextRechts=:,
labelbreite=1cm,
%labeleinzug=2cm,
%abstandEintrag=1,
%newpage=ja,
%pnumwidth=20mm,
%dotsep=1000,
%tocrmarg=4.5cm,
%abstandVerzeichnis=-10mm
]

%
% Tabellenverzeichnis
%
\seKonfigurationTab[
%verzeichnisname=Liste der Tabellen,
labeltextLinks=,
%labeltextRechts=:,
labelbreite=1cm,
%labeleinzug=0.5cm,
%abstandEintrag=1,
%newpage=ja,
%pnumwidth=20mm,
%dotsep=1000,
%tocrmarg=4.5cm,
%abstandVerzeichnis=-10mm
]

%
% Abk\"urzungsverzeichnis
%
\seKonfigurationAbk[
%verzeichnisname=Liste der Abk\"urzungen,
%labelbreite=3cm,
%labeleinzug=0.5cm,
%abstandEintrag=1,
%newpage=ja,
%abstandVerzeichnis=-10mm
]

%
% Symbolverzeichnis
% 
\seKonfigurationSym[
%verzeichnisname=Liste der Symbole,
%labelbreite=4cm,
%labeleinzug=3.5cm,
%abstandEintrag=1,
%newpage=ja,
%abstandVerzeichnis=-10mm
]

%
% Glossar
%
\seKonfigurationGlo[
%verzeichnisname=Glossar,
%abstandEintrag=0,
]



% (eventuelle) Neudefinition f\"ur die Unter-/\"Uberschriften von Abbildungen, Tabellen und Listings
%
%
%\renewcommand{\seCaptionNameAbbildung}{Abb.}
%\renewcommand{\seCaptionNameTabelle}{Tab.}
%\renewcommand{\seCaptionNameProgramm}{Prg.}


% % (eventuelle) Neudefinition f\"ur Querverweise innerhalb des Textes
%
%
%
%\renewcommand{\seQuerverweisSeite}{Seite}
%\renewcommand{\seQuerverweisAbbildung}{Abb.}
%\renewcommand{\seQuerverweisTabelle}{Tab.}
%\renewcommand{\seQuerverweisProgramm}{Prg.}
%\renewcommand{\seQuerverweisGleichung}{Gl.}
%
\renewcommand{\seQuerverweisChapter}{Kapitel}
\renewcommand{\seQuerverweisSection}{Kapitel}
\renewcommand{\seQuerverweisSubsection}{Kapitel}
\renewcommand{\seQuerverweisSubsubsection}{Kapitel}
\renewcommand{\seQuerverweisParagraph}{Kapitel}


%
% Kommandos f\"ur die Konfiguration von URL-Eintr\"agen im Literaturverzeichnis
%
\renewcommand*{\biburlprefix}{\jblangle{}URL: }
\renewcommand*{\biburlsuffix}{\jbrangle{}}
\renewcommand*{\bibbudcsep}{ -- }
\AddTo\bibsgerman{\renewcommand*{\urldatecomment}{Zugriff am }}


