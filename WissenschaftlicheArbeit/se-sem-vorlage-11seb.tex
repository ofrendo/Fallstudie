% Konfigurationsdatei f\"ur die Pfaddefinitionen einlesen
\input{se-wa-pfade}
%
%
% Festlegung der Sprache: 
\newcommand{\seWaSprache}{deutsch}
%\newcommand{\seWaSprache}{englisch}


%
% Einlesen der .sty-Dateien
%
%%  se-wa-input-styles-v095.tex
%
%  Joerg Baumgart 01.08.2011
%
%  Zusammenfassung und Konfiguration wichtiger Styles f\"ur die 
%  Erzeugung von Seminar-, Projekt- und Bachelorarbeiten
%
%  2012-03-12: auf Version 0.94 umgestellt
%
%
% 2012-12-13: auf Version 0.95 umgestellt
%                     Sprachoptionen englisch/deutsch zusammengef\"uhrt
%                     bchart.sty hinzugenommen
%                 
%


\documentclass[12pt,BCOR=10mm,headinclude=on,footinclude=off,bibliography=totoc]{scrreprt}
\usepackage[T1]{fontenc}
\usepackage[latin1]{inputenc}
\usepackage{ifthen}
% 2012-12-13
\ifthenelse{\equal{\seWaSprache}{deutsch}}{% Deutsche Einstellungen
\usepackage[ngerman]{babel}% 
}{% Englische Einstellungen
\usepackage[english]{babel}% 
}

\usepackage{lmodern}

\usepackage{tikz} % Graphikpaket, das zu pdfLaTeX kompatibel ist
\usepackage{xkeyval} % Definition von Kommandos mit mehreren optionalen Argumenten
\usepackage{listings} % Formatierung von Programmlistings
\usepackage{graphicx} % Einbinden von Graphiken
\usepackage{color}
\usepackage{\seWaPathSty/slashbox} % Diagonalen in Tabellenfeldern
\usepackage{framed} % Erzeugung schwarzer Linien am linken Rand zur Hervorhebung von Textteilen
\usepackage{caption} % Korrektes Setzen einer mehrzeiligen float-Unterschrift bei neu definierten float-Umgebungen
\usepackage{floatrow}
% 2012-12-13
\usepackage{\seWaPathSty/bchart} % Kommandos zur Erzeugung von Balkendiagrammen


% Es wird jeweils die sty-Datei importiert und entsprechende Konfigurationseinstellungen werden vorgenommen

\usepackage{\seWaPathSty/se-jb-scrpage2} % Formatierung der Kopf- und Fu{\ss}zeilen
\usepackage{\seWaPathSty/se-jb-footmisc}    % Fussnoten besser formatieren

\usepackage{\seWaPathSty/se-jb-glossaries-v094} % Abk\"urzungsverzeichnis, Symbolverzeichnis, Glossar
   
\usepackage{\seWaPathSty/se-jb-floatrow}    % Definition und Konfiguration von float-Umgebungen (figure, table, die neue programm-Umgebung)
% Achtung: se-jb-varioref muss nach se-jb-floatrow importiert werden; 
% andernfalls ist der counter programm f\"ur die labelformat-Anweisung noch nicht definiert   
\usepackage{\seWaPathSty/se-jb-varioref}   % Definition von Querverweisen
\usepackage{\seWaPathSty/se-jb-chngcntr}   % Kapitelweise oder globale Nummerierung von Abbildungen etc.
   
\usepackage{\seWaPathSty/se-jb-listen} % Definition neuer, besser formatierter Listen
\usepackage{\seWaPathSty/se-jb-kommandos-v095} % neue Kommandos f\"ur Seminar-, Projekt- und Bachelorarbeiten
% 2012-12-13
\ifthenelse{\equal{\seWaSprache}{englisch}}{\usepackage{\seWaPathSty/se-jb-kommandos-englisch}}{}

%  se-wa-input-styles-v095.tex
%
%  Joerg Baumgart 01.08.2011
%
%  Zusammenfassung und Konfiguration wichtiger Styles f\"ur die 
%  Erzeugung von Seminar-, Projekt- und Bachelorarbeiten
%
%  2012-03-12: auf Version 0.94 umgestellt
%
%
% 2012-12-13: auf Version 0.95 umgestellt
%                     Sprachoptionen englisch/deutsch zusammengef\"uhrt
%                     bchart.sty hinzugenommen
%                 
%


\documentclass[12pt,BCOR=10mm,headinclude=on,footinclude=off,bibliography=totoc]{scrreprt}
\usepackage[T1]{fontenc}
\usepackage[latin1]{inputenc}
\usepackage{ifthen}
% 2012-12-13
\ifthenelse{\equal{\seWaSprache}{deutsch}}{% Deutsche Einstellungen
\usepackage[ngerman]{babel}% 
}{% Englische Einstellungen
\usepackage[english]{babel}% 
}

\usepackage{lmodern}

\usepackage{tikz} % Graphikpaket, das zu pdfLaTeX kompatibel ist
\usepackage{xkeyval} % Definition von Kommandos mit mehreren optionalen Argumenten
\usepackage{listings} % Formatierung von Programmlistings
\usepackage{graphicx} % Einbinden von Graphiken
\usepackage{color}
\usepackage{\seWaPathSty/slashbox} % Diagonalen in Tabellenfeldern
\usepackage{framed} % Erzeugung schwarzer Linien am linken Rand zur Hervorhebung von Textteilen
\usepackage{caption} % Korrektes Setzen einer mehrzeiligen float-Unterschrift bei neu definierten float-Umgebungen
\usepackage{floatrow}
% 2012-12-13
\usepackage{\seWaPathSty/bchart} % Kommandos zur Erzeugung von Balkendiagrammen


% Es wird jeweils die sty-Datei importiert und entsprechende Konfigurationseinstellungen werden vorgenommen

\usepackage{\seWaPathSty/se-jb-scrpage2} % Formatierung der Kopf- und Fu{\ss}zeilen
\usepackage{\seWaPathSty/se-jb-footmisc}    % Fussnoten besser formatieren

\usepackage{\seWaPathSty/se-jb-glossaries-v094} % Abk\"urzungsverzeichnis, Symbolverzeichnis, Glossar
   
\usepackage{\seWaPathSty/se-jb-floatrow}    % Definition und Konfiguration von float-Umgebungen (figure, table, die neue programm-Umgebung)
% Achtung: se-jb-varioref muss nach se-jb-floatrow importiert werden; 
% andernfalls ist der counter programm f\"ur die labelformat-Anweisung noch nicht definiert   
\usepackage{\seWaPathSty/se-jb-varioref}   % Definition von Querverweisen
\usepackage{\seWaPathSty/se-jb-chngcntr}   % Kapitelweise oder globale Nummerierung von Abbildungen etc.
   
\usepackage{\seWaPathSty/se-jb-listen} % Definition neuer, besser formatierter Listen
\usepackage{\seWaPathSty/se-jb-kommandos-v095} % neue Kommandos f\"ur Seminar-, Projekt- und Bachelorarbeiten
% 2012-12-13
\ifthenelse{\equal{\seWaSprache}{englisch}}{\usepackage{\seWaPathSty/se-jb-kommandos-englisch}}{}


%
% Individuelle Konfiguration des Dokumentes
%
%\input{\seWaPathText/wa-konfiguration}
\input{se-wa-textbausteine-vorlagen/wa-konfiguration}

%
% Definition von Abk\"urzungen, Symbolen und eventuell Glossareintr\"agen
%
%\input{\seWaPathText/wa-abkuerzungen} 
\input{se-wa-textbausteine-vorlagen/wa-abkuerzungen}

\seIstSeminararbeit{}

\newcommand{\version}{0.953}

% 
% Diese Redefinition ist nur f\"ur den Anhang der  
% Vorlage (Hinweise zur Installation und \"Ubersetzung)
% notwendig; f\"ur Ihre Seminar-/Projekt-/Bachelorarbeit spielt sie keine Rolle
%
\renewcommand{\seVorlage}{\jobname}

% Verwendung kleinerer Schriftgroessen f�r die Ueberschriften; sinnvoll bei kurzen Texten.
%
%
\KOMAoption{headings}{small}

% Fuer einen Kapitelanfang wird kein zusaetzlicher vertikaler Abstand erzeugt
%
%
\seNoChapterSkip[-12.25mm]{}

\begin{document}
% Erzeugung des Titelblatts
%
%
%
\seTitelblattSeminararbeit[
%hilfslinien=ja,
%dhbwlogoSkalierung=0.5,
%dhbwlogoDeltaX=2.4,
%dhbwlogoDeltaY=-10,
studiengang=\seWirtschaftsinformatik,
studienrichtung=\seSoftwareEngineering,
thema=Fallstudie,
verfasser={J�rn H�bner , Patrick Knerr, Marco Riege, Matthias Liedtke, Oliver
Frendo},
%verfasserin= Melanie Musterfrau,
kurs=WWI\,12\,SE\,B,
firma=SAP,
% Da im Text ein Komma enthalten ist, muss der Text eingeklammert werden
%studiengangsleiterin=,
studiengangsleiter=Prof. Dr. Herr Holey,
modul=Umsetzung von Methoden der Wirtschaftsinformatik,
lehrveranstaltung=Fallstudie,
%dozentin=,
dozent=Gregor Tielsch
]


% Erzeugung der englischen Kurzfassung (Abstract); Verfasser, Firma und Thema werden automatisch \"ubernommen
%
% Der optionale Parameter kann verwendet werden, um f\"ur das Thema der Arbeit eine 
% andere Formatierung vorzunehmen; das sollte in der Regel nicht erforderlich sein;
% ausserdem besteht die Gefahr inkonsistenter Titel auf dem Titelblatt und in der 
% Kurzfassung
%
%
% Achtung: Das Kommando erzeugt nur dann eine Ausgabe, wenn \seWaSprache den Wert englisch besitzt
%
%
%\seAbstract{} % dieses Kommando sollte standardm\"assig verwendet werden

%\seAbstract[\LaTeX-Vorlage zur Anfertigung \seThemaWaArbeit{} (Version \version{})]



% Erzeugung der Kurzfassung; Verfasser, Firma und Thema werden automatisch \"ubernommen
%
% Der optionale Parameter kann verwendet werden, um f\"ur das Thema der Arbeit eine 
% andere Formatierung vorzunehmen; das sollte in der Regel nicht erforderlich sein;
% ausserdem besteht die Gefahr inkonsistenter Titel auf dem Titelblatt und in der 
% Kurzfassung
%
%\seKurzfassung{} % dieses Kommando sollte standardm\"assig verwendet werden

%\seKurzfassung[\LaTeX-Vorlage zur Anfertigung einer Seminararbeit (Version
% \version{})]


% Beispiel f\"ur ein Kapitel, dass vor dem Einleitungskapitel kommt, z. B. ein Vorwort oder eine Danksagung
%\seKapitelVorEinleitung{Vorwort}
%
%Muss jetzt wirklich nicht sein, aber wenn Sie unbedingt (z. B.) Ihrem Haustier f\"ur die Unterst\"utzung bei 
%der Anfertigung der Projektarbeit danken wollen ...; vgl. auch das Dokument \textsl{Empfehlungen und 
%Hinweise zur Anfertigung der zweiten Projektarbeit}


% 2012-02-06 Inhaltsverzeichnis muss vor den weiteren Verzeichnisses kommen
%
%
% Ausgabe des Inhaltsverzeichnisses
%
%
\seInhaltsverzeichnis[%
einrueckung=ja,
gliederungsebenen=4
]

%
% 
% Wenn die Verzeichnisse (ohne Seitenvorschub) nach dem Inhaltsverzeichnis
% kommen sollen, sind die beiden folgenden Kommandos zu verwenden
%
% Ein neues Kapitel beginnt nicht auf einer neuen Seite
%
%
%\seChaptersWithoutNewpage{}

% Erzeugung eines vertikalen Abstands nach diesem Kapitel
%
%\seChapterEndSkip{}


% Ausgabe der verschiedenen Verzeichnisse
% abk: Abk\"urzungsverzeichnis
% sym: Symbolverzeichnis
% abb: Abbildungsverzeichnis
% tab: Tabellenverzeichnis
% prg: Listingverzeichnis
%
%
% Achtung: Abk\"urzungs- und Symbolverzeichnis werden nur ausgegeben, wenn mindest ein Symbol bzw. 
%                mindestens eine Abk\"urzung in der Arbeit verwendet wurden
%
%
% gliederungsebene:
% -- section: die Verzeichnisse werden einem Kapitel "Verzeichnisse" untergliedert
% -- chapter: die Verzeichnisse sind jeweils eigene Kapitel
% imInhaltsverzeichnis: ja/nein -- Sollen die Verzeichnisse im Inhaltsverzeichnis enthalten sein?
\seVerzeichnisse[gliederungsebene=section,imInhaltsverzeichnis=ja]{abk}{sym}{abb}{tab}{prg}


\seChaptersNewpage{}

\chapter{Einleitung}\pagenumbering{arabic}
%Matthias 
\section{Problemstellung und Zielsetzung Matthias}
Aufgabe der vorliegenden Arbeit ist es, ein computergest�tztes
Unternehmensplanspiel zu entwickeln, wobei m�glichst viele erworbene Methoden
und Kompetenzen des Software Engineerings Anwendung finden sollen.\\
Ein Unternehmensplanspiel bezeichnet dabei \glqq eine modellhafte Simulation von
Unternehmensprozessen.\grqq

Als Rahmenbedingungen wurden festgesetzt, dass eine objektorientierte Analyse
(OOA) und ein objektorientierter Entwurf (OOD) durchgef�hrt werden. Dar�ber
hinaus soll ein Prototyp der Benutzeroberfl�che (UI) erstellt werden, sowie ein
in Java implementiertes Fachkonzept. \\
Die Funktionsf�higkeit des Fachkonzepts ist durch die Verwendung von geeigneten
UnitTests sicherzustellen.

Der Projektplan sieht vor, dass die Entwicklung im Wesentlichen in mindestens
zwei Iterationen unterteilt ist, wobei nach jeder Iteration in einer
Pr�sentation die bisherigen Ergebnisse dargestellt werden sollen.


% Matthias
\section{Vorgehen Matthias}
Die ersten zwei Wochen der Planungsphase wurden darauf verwendet, einen
geeigneten Markt zu finden und die in Frage kommenden M�rkte auf ihr Potenzial
hin zu analysieren. Da diese Entscheidung ma�gebend auf alle folgenden Aufgaben
Einfluss hat und letztlich auch auf die Realisierbarkeit des
Unternehmensplanspiels, haben wir diese Zeit genutzt, um gemeinsam Brainstorm zu
betreiben, das F�r und Wider verschiedener M�rkte abzuw�gen und grundlegende
Konzepte der Spielmechanik zu evaluieren.\\
Die Kunst lag hierbei in der Konzentration auf die Ideenfindung, ohne zu sehr
ins Detail zu gehen oder sich in Themen der Analyse oder des Entwurfs zu
begeben.

Nachdem sowohl der abzubildende Markt als auch die grundlegenden Spielprinzipien
erarbeitet wurden, ging es in die Analysephase.\\
Zu Beginn der Analysephase haben wir uns �berlegt, welche Anforderungen zwingend
Bestandteil unseres Programms werden sollen, und welche optional sind und das
Spielerlebnis zwar erweitern, f�r die Funktionsf�higkeit aber nicht essentiell
sind.\\
Anschlie�end haben wir auf Basis dieser Anforderungen einen allerersten
Prototypen eines Designs f�r das Spiel entworfen, wobei es hier nur um die grobe
Darstellung der spielrelevanten Informationen und Mechanismen ging, nicht jedoch
um Einzelheiten wie Anordnungen von Buttons oder Farbschemata.

Danach haben wir die Analyse in die Pakete Server, Client-Server und
Client (Company) unterteilt. Das UI wurde abgesehen von einem ersten Prototyp
aus Zeitgr�nden und der nicht zwingenden Anforderung einer Implementierung in
der Analysephase vernachl�ssigt. Somit beschr�nkte sich unsere Analysephase im
Wesentlichen auf das Fachkonzept und die Spielmechaniken.

Bei der Analyse des Fachkonzepts stand das dabei entworfene Klassendiagramm im
Vordergrund, wobei auch ein Use Case Diagram entworfen wurde.

\seChapterEndSkip{}

\seChaptersWithoutNewpage{}

\newpage
\chapter{Grundlagen}
%Patrick
\section{BWL Hintergrund Patrick}\label{patrick:bwl}

\newpage
\chapter{Analyse}
% Matthias
\section{Spielideen}
Bei der Spielidee war es uns wichtig, auf der einen Seite einen interessanten
und wirtschaftlich herausfordernden Markt darzustellen und auf der anderen Seite
dem Spieler das Gef�hl zu geben, ein Spiel zu spielen.\\
W�hrend der erste Punkt eine gewisse Recherche m�glicher M�rkte beinhaltet, geht
es beim zweiten vielmehr darum, typische Mechaniken eines Spiels zu
implementieren.

Im Folgenden werden die beiden Punkte genauer untersucht:
\subsection{Der Markt des Spiels: Strommarkt}
Bei der Analyse unterschiedlicher M�rkte haben sich einige Eigenschaften
herausgestellt, die f�r die Realisierung eines m�glichen Spiels entscheidend
waren:

\begin{itemize}
  \item \textbf{Oligopol}: Da jeder Spieler die M�glichkeit haben soll,
  den Markt zu beeinflussen, sollte es kein Massenmarkt mit hunderten Anbietern
  sein. Tats�chlich ist es f�r einen Spieler ein weitaus intensiveres
  Spielerlebnis, wenn er sich selbst als relevanten --- wenn nicht sogar
  essentiellen --- Marktteilnehmer sieht.
  \item \textbf{Produktionssektor}: Auch wenn es gewiss einen gewissen Charme
  h�tte, ein Unternehmen des Dienstleistungssektors abzubilden, fiel bei diesem
  Spiel die Entscheidung, das Unternehmen im prim�ren Sektor anzusiedeln. Das
  hat den Vorteil, dass sich elementare marktbegrenzende Mechanismen wie
  Ressourcenknappheit oder begrenzte Nachfrage simulieren lassen.
  \item \textbf{Differenzierter Markt / differenzierte Produktion}: Ein weiterer
  Fokus lag darauf, dass dem Spieler Handlungsfreiheiten gegeben werden sollen,
  die nicht nur darin bestehen, die Kosten f�r einzelne Abteilungen und den
  Preis f�r das Endprodukt einzustellen, sondern ihm auch eine gewisse
  Diversifikation und damit die M�glichkeit einer Ausarbeitung von verschiedenen
  Strategien zur Hand zu geben. Dies lie�e sich einerseits durch einen Markt mit
  mehreren klar differenzierbaren Produkten (z.B. Smartphone und Tablet) als
  auch durch eine mehrstufige Produktionskette realisieren. (Mit der
  M�glichkeit, nur manche Abschnitte der Wertsch�pfungskette selbst abzubilden.)
\end{itemize}

Als interessantester und vielversprechendster Markt hat sich schlie�lich der
Energiemarkt --- in diesem Fall der Strommarkt --- herauskristallisiert. Das
Oligopol ist hier insofern gegeben, als dass die vier gr��ten Stromproduzenten
in Deutschland (E.ON, RWE, EnBW und Vattenfall) einen Marktanteil von 80\%
abdecken.

Um nun auch ein Mindestma� an Handlungsfreiheit zu gew�hrleisten, wurde beim
Spielkonzept nicht nur der Wertsch�pfungsaspekt der direkten Stromproduktion
durch Kraftwerke betrachtet, sondern auch die Gewinnung der daf�r erforderlichen
Ressourcen. Es liegt am Spieler, zu entscheiden, ob es am lukrativsten ist, die
erforderlichen Ressourcen in Minen abzubauen oder diese direkt am Markt zu
kaufen. Dar�ber hinaus stehen dem Spieler durch die Wahl zwischen diversen
fossilen Brennstoffen (Gas, Uran (Kernenergie), Kohle) und erneuerbaren Energien
(Wind, Wasser, Solar) einige Aspekte der unternehmerischen Entscheidungsfindung
zur Verf�gung.

\begin{figure}[H]
\centering
\centering
\includegraphics[width=0.8\textwidth]{se-wa-jpg/produktionskette}
\caption{Produktionskette}
\label{Produktionskette}
\end{figure}

Ein weiterer Aspekt ist, dass der Kauf von Kraftwerken immer eine hohe
Investition darstellt und damit eine hohe Tragweite hat, der Spieler also nicht
dazu gezwungen ist, jede Runde viele Investitionen zu t�tigen, um sein Kapital
sinnvoll binden zu k�nnen. Im Gegenteil: Der Spieler wird dazu forciert,
einige wenige (im Idealfall wohl �berlegte) Entscheidungen zu treffen, die entscheidend
f�r den weiteren Spielverlauf sein k�nnen.
\subsection{Das Spielprinzip}
Neben dem zu realisierenden Markt lag ein gro�er Fokus auf der Modellierung
dieses Marktes in einem geeigneten Modell, das auf der einen Seite sowohl den
Markt in seinen Grundz�gen abbildet, auf der anderen Seite aber auch einen hohen
Wert auf Spielspa� und Spielbarkeit legt.\\
Dazu sollte das Unternehmen vor allem nicht nur in Zahlen greifbar sein.
Investitionen (z.B. in ein neues Kraftwerk) sollen dem Spieler auch ein
Erfolgserlebnis bescheren, das ihm am besten auch visuell schmackhaft gemacht
wird.

F�r diese Ziele musste nun eine geeignete Abbildung in ein Spielkonzept gefunden
werden. Inspiriert vom Brettspielklassiker "`Die Siedler von Catan"' und
dem rundenbasierten Globalstrategiespiel "`Sid Meier's Civilization V"'
kam die Idee eines Sechseckrasters als "`Spielfeld"' auf. Dies
erm�glicht eine Visualisierung des Grundgeschehens und zus�tzliche
Handlungsfreiheiten, da der Spieler entscheiden kann, welches Feld er wo kauft.

\begin{figure}[H]
\centering
\centering
\includegraphics[width=1.0\textwidth]{se-wa-jpg/Sechseckraster}
\caption{Sechseckraster}
\label{Sechseckraster}
\end{figure}

Um die Spielmechaniken nicht noch weiter zu verkomplizieren und dem Spieler
einen gewissen �berblick zu lassen, wurde auf ein rundenbasiertes Spielmodell
gesetzt.

%Matthias
\section{Use Case Diagramm}

\begin{figure}[H]
\centering
\centering
\includegraphics[width=0.9\textwidth]{se-wa-jpg/usecase}
\caption{Use Case Diagramm}
\label{Use Case Diagramm}
\end{figure}

Der Spieler hat im System des Unternehmensplanspiels folgende Optionen:
\begin{itemize}
  \item Investment t�tigen\\
  Hierunter fallen die folgenden Aktionen:
  	\begin{itemize}
  	  \item Mine bauen
  	  \item Kraftwerk bauen
  	  \item Grundst�ck kaufen
  	\end{itemize}
  \item Unternehmensergebnisse einsehen\\
  Der Spieler kann sich sowohl das Warenlager, den Kassenbestand, die Bilanz und
  die GuV einsehen.
  \item Vertrag mit Stadt abschlie�en\\
  Der Spieler kann mit St�dten einen Vertrag abschlie�en, um die Stadt mit
  Energie zu versorgen.
  \item Handel t�tigen\\
  Der Spieler kann die Rohstoffe Uran, Gas und Kohle zu Festpreisen kaufen oder
  verkaufen. Dar�ber hinaus kann er an der Energieb�rse �bersch�ssige Energie
  verkaufen oder fehlende Energie beziehen.
  \item Kredit aufnehmen
  Zur Finanzierung von Investitionen kann der Spieler einen Kredit aufnehmen.
  Die Konditionen sind dabei fest, der Spieler kann jedoch aus unterschiedlichen
  Krediten ausw�hlen.
\end{itemize}
%Olli
\section{Architektur}
In diesem Kapitel wird die Software Architektur des Projektes vorgestellt. Das
Spiel wurde nach dem ``MVC'' (Model View Controller) Konzept entwickelt. Das bedeutet,
dass die Klassen, die das Modell und dessen Daten darstellen, sowohl das View
als auch den Controller nicht kennen. Das View, das in diesem Fall die
graphische Benutzeroberfl�che ist kennt hingegen zwar das Modell, den
Controller aber nicht. Diese Trennung ist wichtig, damit die Datenhaltung und
Logik seperat von der graphischen Benutzeroberfl�che entwickelt werden kann.

Der Controller kennt beide anderen Ebenenen und
verwaltet diese --- es sorgt beispielsweise daf�r, dass die Benutzeroberfl�che
immer die aktuellen Daten anzeigt. In unserem Projekt wird der Controller durch
eine einzelne Klasse gebildet, die das Singleton Pattern implementiert. Dadurch
kann jede Klasse statisch auf den Controller zugreifen, ohne eine Referenz zu
ihm speichern zu m�ssen --- und trotzdem kann der Controller objektorientiert
entwickelt werden. %Das hier eher = Implementierung?

%MVC Bild
\begin{figure}[H]
\centering
\includegraphics[width=0.5\textwidth]{se-wa-jpg/mvc}
\caption{Model View Controller Konzept}
\label{MVC Konzept}
\end{figure}


Es wurde fr�h entschieden, dass das Spiel nicht nur auf einem Computer
laufen soll (``Hotseat''), sondern dass das Spiel von einem Server kontrolliert
wird und die Daten an Clients, die auf unterschiedlichen Computern
laufen k�nnen, verschickt werden --- somit k�nnen die Spieler gleichzeitig
spielen und haben keine langen Wartephasen. Die Logik des Spiels und die
Datenhaltung befinden sich verteilt auf dem Server und auf den einzelnen
Clients. Der Server verwaltet die wichtigsten Daten, die allen Clients bekannt
sein m�ssen (beispielsweise das Spielfeld und Informationen dar�ber) w�hrend auf
den Clients zum Beispiel die finanziellen Daten wie die Bilanz gehalten
werden.

Des Weiteren wurde keine persistente Speicherung der Spieldaten vorgesehen.
Sobald der Server also nicht mehr l�uft sind die Daten des aktuellen Spiels
verloren. Dies reduziert die Komplexit�t des Modells, denn es muss beim
Initialisieren des Spiels keine Daten laden und beim Beenden des Spiels keine
Daten speichern.


\newpage
\section{OOA-Klassenmodell}

Im Folgenden wird unser, auf Basis der Anforderungen unseres Planspiels,
erstelltes Klassendiagramm der Analysephase aus \ref{OOA-Client} und
\ref{OOA-Server} auf \pageref{OOA-Client} und \pageref{OOA-Server} n�her
erl�utert. 

Das Klassendiagramm ist in zwei H�lften aufgeteilt. In dem ersten Teil werden
die Klassen f�r den Clienten und die Klassen, die f�r die r�ndliche Kommunikation
zwischen Server und Client genutzt werden, abgebildet. Der zweite Teil enth�lt
die Klassen des Servers und Klassen und Interfaces, die f�r die
au�erordentliche Kommunikation zwischen Server und Client vorgesehen sind.

Das Herzst�ck des Unternehmens, dass auf Client-Seite vorgesehen ist, ist die
'Company' -Klasse. In ihr werden alle Entscheidungen des Spielers bearbeitet
und sie enth�lt alle Informationen, die der Spieler �ber sein Unternehmen
ben�tigt.
Dies sind die Mitarbeiter, die einzelnen Abteilungen und die Beziehungen, die zu
den einzelnen Regionen bestehen.

In den Beziehungen zu den Regionen werden entweder in einer 'ResourceRelation'
der Besitztum dieser Region erlangt und, sollte man schon eine solche Region
besitzen, auch die vorhandenen Geb�ude dieser Region gespeichert. Die
'CityRelation' enth�lt alle Informationen �ber die Bewohner einer Stadt und mit
dem 'Contract' auch �ber die Kunden des Unternehmens.

Die Abteilungen des Unternehmens umfassen das 'Warehouse', das
'InvestmentManagement', den 'Research', die 'Finances' und das 'Marketing'. In
dem Warehouse werden die Rohstoffe gelagert, die in Minen produziert
werden und die f�r die Produktion von Strom in den Kraftwerken ben�tigt werden.
Auch gehandelter Rohstoffe werden hier entnommen oder eingelagert.\\
Das InvestmentMangement beinhaltet alle Geb�ude (Minen und Kraftwerke) und alle
Grundst�cke (Regionen), die das Unternehmen besitzt. Hier werden neue Geb�ude
hinzugef�gt, Abschreibungen berechnet und Produktionsmengen eingestellt und
ausgelesen. Zudem besitzt jedes Kraftwerk mehrere 'PowerStationRelation', die
beinhaltet wieviel Energie das Kraftwerk den einzelnen, umliegenden St�dten
liefert. Diese Beziehung von einem Kraftwerk zu den St�dten ist vorgesehen,
damit die gewollte, maximale Lieferentfernung von drei Feldern nicht
�berschritten wird.\\
Die 'Finances' sind vorgesehen, um alle vier Quartale eine Bilanz und eine
Gewinn und Verlustrechnung aufzustellen. Alle Einnahmen und Ausgaben werden hier
eingespeichert und aufbereitet.\\
Das 'Marketing' ist vorgesehen um die Beliebtheit und die Bekanntheit
bei den Kunden zu beeinflussen.\\
Der letzte Bereich, der 'Research', ist vorgesehen um m�glicherweise eine aktive
Forschung einbauen zu k�nnen. Auf Grund von anderen Priorit�ten ist dieser aber 
nicht weiter modelliert und auch nicht in das entg�ltige Planspiel aufgenommen
worden.

Die Verbindung zwischen Client und Server wird aufgebaut indem sich der Client
�ber die 'Client'-Klasse mit der 'Server'-Klasse auf Serverseite verbindet. Dort
wird die Verbindung akzeptiert und nach dem 'Thread-per-Connection'-Prinzip ein,
f�r jeden Clienten seperater Thread der Klasse 'Connection' erstellt.

Die anschlie�ende Kommunikation wird mit Objekten, die jeweils das
'Messagable'-Interface implementieren durchgef�hrt. So kann jede beliebige
Klasse �bertragen werden. Der Empf�nger des Objektes kann nun den MessageType
des Objektes abfragen und wei� somit, wie er mit dem Objekt weiter verfahren
soll. Hierf�r sind spezielle Typen f�r jede unterschiedliche Nachricht, die
kommuniziert werden soll, in zwei Enumerates definiert.

�ber den Server werden jede Runde, alle Entscheidungen der Spieler abgewickelt,
die nicht nur den Spieler selbst, sondern auch andere Spieler betreffen. Vor
allem sind dies, Grundst�cksgebote und -k�ufe und Vertragseinstellungen mit
einer Stadt, nach denen die Kunden des Spielers bestimmt werden. Auch bereits
gebaute Geb�ude werden dem Server mitgeteilt, so dass sie f�r jeden Spieler
ersichtlich werden.

Diese �nderungen werden dem Clienten zu jedem Rundenbeginn �ber das
'Map'-Objekt und die 'CityRelation' mitgeteilt.




\begin{figure}
\centering
\centering
\hspace*{-30mm}
\includegraphics[width=1.3\textwidth]{se-wa-jpg/Client}
\caption{Klassendiagramm Teil 1}
\label{OOA-Client}
\end{figure}
%Die Grafik in Abbildung 
%\ref{labelname} auf Seite \pageref{labelname} ..
\begin{figure}
\centering
\centering
\includegraphics[width=1.1\textwidth]{se-wa-jpg/Server}
\caption{Klassendiagramm Teil 2}
\label{OOA-Server}
\end{figure} %OOA
%Olli
\section{Optimierung}\label{Opt}
In diesem Teil der Arbeit wird die Optimierung der Kraftwerksverteilung
erl�utert. Dies wird auf der Seite des Clients ausgef�hrt --- dadurch wird die
Komplexit�t des Algorithmus reduziert, da die Daten des Servers nicht mit denen 
des Clients synchronisiert werden m�ssen. Aus Spielgr�nden wurde entschieden, 
dass sich der Spieler nicht selbst um die Optimierung der Kraftwerke k�mmern 
muss, sondern dass die Optimierung mithilfe des Simplexalgorithmus durchgef�hrt 
wird, da die Kraftwerksverteilung durch ein lineares Optimierungsproblem
formuliert werden kann (\seCite{vgl.}{}{holey}).

Zun�chst stellt dies das folgende Problem dar: \\
$k$ Kraftwerke mit einer Produktion von $k_i$ haben je $x$ Verbindungen zu $s$
St�dten mit je $y$ Verbindungen, einem Preis $p$ und einem Bedarf $n$. Dabei wird eine Verbindung nur gebildet,
wenn der Spieler mit der jeweiligen Stadt einen Vertrag besitzt und das
Kraftwerk  h�chstens drei Felder von der Stadt entfernt ist. So k�nnte 
beispielsweise folgendes Optimierungsproblem entstehen:

%Bild Optimierung
\begin{figure}[H]
\centering
\includegraphics[width=0.85\textwidth]{se-wa-jpg/optimizing}
\caption{Beispiel Optimierungsproblem}
\label{Beispielproblem}
\end{figure}

Nach dem Simplexalgorithmus m�ssen f�r dieses Problem nun die Zielfunktion und
die Nebenbedingungen aufgestellt werden. Die Zielfunktion $z$ ist durch die 
Maximierung des Umsatzes gegeben. Dabei ist $p_i$ der Preis der Stadt $i$,
$k_j$ die maximale Produktion des Kraftwerkes $j$ und $y_j$ die Verbindung
zwischen der Stadt und dem Kraftwerk.

\begin{align*}
z &= \sum\limits_{i=1}^s p_i * \sum\limits_{j=1}^y y_j * k_j \\
  &= 0.39 (x_{11} * 30 + x_{21} * 20) + 0.35 (x_{22} * 20 + x_{31} * 40)
\end{align*}

Als n�chstes wird f�r jedes Kraftwerk eine Nebenbedingung erstellt. Die
Verteilung der Produktion eines Kraftwerkes ist durch mehrere
Variablen $x_i$ gegeben (pro Kraftwerkverbindung eine).
Addiert d�rfen diese h�chstens $1$ sein, damit das Kraftwerk nicht mehr auf die
Verbindungen verteilt, als es eigentlich maximal produzieren kann:

\begin{align*}
g: &\sum\limits_{i=1}^x x_i &\leq 1 \\
g_1: &x_{11} &\leq 1 \\
g_2: &x_{21} + x_{22} &\leq 1 \\
g_3: &x_{31} &\leq 1
\end{align*}

Anschlie�end wird f�r jede Stadt eine Nebenbedingung erstellt, in der der
maximale Bedarf $n$ der Stadt gepr�ft wird. $y$ ist hier die Anzahl der Verbindungen der
Stadt und $k_j$ die Produktion des Kraftwerkes $j$:

\begin{align*}
g: \sum\limits_{j=1}^y y_j * k_j \leq n \\
g_4: 30x_{11} + 20x_{21} \leq 50 \\
g_5: 20x_{22} + 40x_{31} \leq 60
\end{align*}

Danach werden die Zielfunktion und die Nebenbedingungen in die Normalform
gebracht. In diesem Fall werden in den Nebenbedingungen nur Schl�pfvariablen
angelegt und als Startl�sung die Kraftwerksverbindungsvariablen $x_i = 0$
gesetzt. Darauffolgend werden nach dem Simplexalgorithmus die Daten in eine
Matrix eingef�gt:

\begin{align*}
\begin{array}{ c | c | c | c | c | c | c | c | c | c | c }
BV & x_{11} & x_{21} & x_{22} & x_{31} & s_1 & s_2 & s_3 & s_4 & s_5 & b \\
\hline
s_1 & 1 & 0 & 0 & 0 & 1 & 0 & 0 & 0 & 0 & 1 \\ 
s_2 & 0 & 1 & 1 & 0 & 0 & 1 & 0 & 0 & 0 & 1 \\
s_3 & 0 & 0 & 0 & 1 & 0 & 0 & 1 & 0 & 0 & 1 \\
s_4 & 30 & 20 & 0 & 0 & 0 & 0 & 0 & 1 & 0 & 50 \\
s_5 & 0 & 0 & 20 & 40 & 0 & 0 & 0 & 0 & 1 & 60 \\
\hline
z & -0.39*30 & -0.39*20 & -0.35*20 & -0.35*40 & 0 & 0 & 0 & 0 & 0 & 0
\end{array}
\end{align*}

Nach mehreren Iterationsschritten zeigt sich folgende L�sung als optimal:

\begin{align*}
x_{11} &= 1 & s_1 &= 0 \\
x_{21} &= 1 & s_2 &= 0 \\
x_{22} &= 0 & s_3 &= 0 \\
x_{31} &= 1 & s_4 &= 0 \\
& & s_5 &= 20
\end{align*}

Die Stadt $s_1$ wird aufgrund des h�heren Preises vom Kraftwerk $k_2$ bevorzugt
und wird deswegen �ber die Verbindung $x_{21} = 1$ beliefert. $x_{22} = 0$
bedeutet, dass $k_2$ zwar an Stadt $s_2$ liefern k�nnte, es aber nicht macht. Die
Schl�pfvariablen $s_{1\ldots3} = 0$ bedeutet, dass alle Kraftwerke ihre
komplette Produktion nutzen. Zuletzt bedeutet $s_4 = 0$, dass der Bedarf von
Stadt $s_1$ komplett gef�llt wird und $s_5 = 20$, dass der Stadt $s_2$ noch
$20$ Einheiten fehlen.




%Olli
\section{Vertr�ge}\label{olli:vertraege}
Ein wichtiger Teil des Spieles ist f�r den Benutzer das Abschlie�en von
Vertr�gen mit St�dten. Dabei soll zwischen den Spielern eine Konkurrenz
entstehen, um einen Bezug zur Realit�t herzustellen. Dieser
Algorithmus wird jede Runde erneut durchlaufen, damit eine Preiskonkurrenz
zwischen Spielern entsteht --- zudem kann sich dadurch die Anzahl der Kunden
�ndern, ohne dass der Spieler Werte �ndert. Bevor der Algorithmus durchgef�hrt
wird, werden alle bestehenden Vertr�ge mit der Stadt nach ihrem Preis
aufsteigend geordnet. Dies hat zur Auswirkung, dass die Preiswahl des Spielers
ausschlaggebend f�r die Anzahl seiner Kunden ist.

Zun�chst wird die Anzahl der Kunden $y_0$ berechnet, die ein Spieler bei dem
durchschnittlichen Preis bei seiner Bekanntheit und Beliebtheit erhalten w�rde. 
$$y_0 = Bekanntheit * Beliebtheit * Population$$
Anschlie�end wird diese Variable der ``Standardkunden'' $y_0$ in eine
quadratische Funktion gegeben. Diese Parabel hat einen Sattelpunkt, der
sich bei $S(x_d/y_0)$ befindet, wobei $x_d$ der Durchschnittspreis der Stadt
representiert. Ist der Preis des Spielers also unter dem Durchschnittspreis,
werden ihm entsprechend mehr und mehr (quadratisch steigend) Kunden zugewiesen 
--- ist der Preis des Spielers unter dem Durchschnittspreis werden ihm umgekehrt
weniger Kunden als $y_0$ zugewiesen. Werden dem Spieler durch diese Funktion $y
= 0$ Kunden zugewiesen, wird der Algorithmus hier abgebrochen und der Spieler  
erh�lt keinen Vertrag mit der Stadt bzw. sein aktueller Vertrag wird gek�ndigt. 

%Bild Paralbel

Durch die Wahl einer quadratischen Funktion sind Preisabweichungen sehr
entscheidend --- anders als bei anderen Produkten und M�rkten ist in dem
Energiemarkt der Preis wichtiger als die Bekanntheit und Beliebtheit einer
Firma. 

Nun werden zwei weitere Bedingungen gepr�ft: \\
Der Spieler kann die maximale Energie festlegen, die er f�r den Vertrag
bereitstellen m�chte. Wird dieser Wert �berschritten, erh�lt er maximal so viele
Kunden, wie er beliefern kann.\\
Zuletzt muss �berpr�ft werden, ob noch genug freie Kunden in der Stadt
vorhanden sind --- es k�nnte also beispielsweise sein, dass der Spieler durch
die Preiswahl anderer Spieler aus diesem Grund keine Kunden mehr erh�lt.

\newpage
\chapter{Implementierung}
\input{parts/projektablauf}
%J�rn
\section{Fachkonzept}
Im \ref{ooa} wurde bereits das Klassendiagramm der Analysephase n�her erl�utert.
Bei der Implementierung auf Basis dieses Klassenmodells haben wir aber noch
einige �nderungen und Einschr�nkungen vorgenommen. Im folgenden Abschnitt werden
diese Unterschiede und dessen Hintergr�nde n�her erl�utert.

Durch Priorit�ten die wir in der Entwurfsphase ges�tzt haben, sind gegen�ber dem
OOA-Diagramm auch einige Bereiche weggefallen:
\begin{itemize}
  \item {\textbf{Marketing}}\\
  Das Marketing und die dazugeh�rige Marktforschung und Werbung haben zeitlich
  nicht mehr in den Plan gepasst.
  \item{\textbf{Mitarbeiter}}\\
  Die Mitarbeiter als einzelnes Objekt zu modellieren erschien uns im Laufe der
  Entwurfsphase in einem Geb�udeintensiven Unternehmen f�r keine Notwendigkeit
  mehr. Deswegen haben wir uns nun dazu entschieden auf eine Mitarbeiteranzahl
  und weiteres zu verzichten und diese vielmehr in die laufenden Kosten der
  einzelnen Kraftwerke mit einflie�en zu lassen.
\end{itemize}



\subsection{Client-Server}


\begin{figure}
\centering
\centering
\includegraphics[width=1.1\textwidth]{se-wa-jpg/OOD-Map}
\caption{Klassendiagramm Kommunikationsklassen}
\label{OOD-Map}
\end{figure}

Die Client-Server Architektur, die bei unserem Planspiel angewandt wird,
erm�glicht es mehreren Spielern ihren Spielzug gleichzeitig zu absolvieren.
Zwar verringert dies die Wartezeiten der Spieler im Gegensatz zum
Hotseat-Modell, allerdings muss bei einem gemeinsamen Server und mehreren
Clienten darauf geachtet werden, dass es zu keinen Dateninkonsistenzen kommt.
An der ``Thread-per-Connection''-Methode hat sich seit der Analysephase (siehe
\ref{ooa}) nichts ge�ndert, weswegen hier darauf nicht n�her eingegangen wird.

In \ref{OOD-Map} ist die Klassenstruktur abgebildet, die alle Daten enth�lt, die
f�r alle Spieler interessant sind und indirekt auch von diesen ge�ndert werden
k�nnen. Hierbei haben wir uns dazu entschlossen nicht alle Entscheidungen des
Spielers dem Server mitzuteilen, sondern einige Eingaben auf Client-Seite
zu bearbeiten, um den Server nicht unn�tig zu belasten und den
Datenverkehr zu vermindern. Zum Beispiel wird das Verm�gen eines Unternehmens
komplett auf Client-Seite bearbeitet, da es f�r die anderen Spieler im Spiel
nicht sichtbar gemacht werden soll.\\
Um Dateninkonsistenz zu vermeiden werden alle �nderungen, die alle Spieler
betreffen, direkt an den Server gesendet, dieser wertet die �nderungen aus
�bertr�gt sie gegebenenfalls in den entsprechenden Objekte. Zu Rundenbeginn
wird nun die gesamt Objektstruktur, die in einem Objekt der 'Map'-Klasse auf
Serverseite gespeichert ist, an alle Clients �bertragen. Dadurch, dass nur der
Server solche �nderungen vornehmen kann und diese auch nicht von verschiedenen
'Connection'-Threads gleichzeitig vorgenommen k�nnen (``Synchronized''), wird
die richtige Datenhaltung sichergestellt.

W�hrend der Analysephase war es angedacht, Vertr�ge mit einer Stadt unabh�ngig
vom Rundenende zu schlie�en. Diese Idee wurde w�hrend der Entwcklung verworfen,
so dass nun jeder Spieler seine Vertragsvorschl�ge dem Server w�hrend der Runde
zusendet, der Server nach der Runde die Kunden jedes Spielers
errechnet und zum n�chsten Rundenanfang diese den Spielern mitteilt.\\
Wie in \ref{OOD-Map} zu sehen ist, haben wir uns deswegen dazu entschieden den
'Contract', der vorher unter dem Unternehmen in dem Bereich der Beziehungen zu
den Regionen angeordnet war, nun der 'Map' und darin der 'CityRegion' zu
unterordnen.\\
Diese �nderung in der Struktur der Fachlogik erlaubt es dem Server weiterhin,
alle Daten, die jede Runde den Spielern zugeschickt werden, in einer einzigen
Objektstruktur zu �bertragen.
Dies erm�glicht zum einen eine einfache Kommunikation vom Server zum Client und
zum anderen kann nun auch der Client alle wichtigen �nderungen der Runde aus
einem einzigen Objekt auslesen, in dem er die neu erhaltene 'Map' mit der
vorhandenen aus der vergangenen Runde vergleicht.

Im folgenden werden die einzelnen Klassen, die vom Server zu dem Client
�bertragen werden, n�her erl�utert:
\begin{itemize}
  \item {\textbf{Klasse Map}}\\
  Die Klasse Map enth�lt zum Einen alle Regionen der Spielkarte, die zu Beginn
  des Spiels vom Server generiert werden und zum Anderen die
  Stromb�rse('EnergyExchange') und den Ressourcenmarkt. Au�erdem werden in ihr
  die Kunden der einzlnen Spieler in den verschiedenen St�dten berechnet.
  \item{\textbf{Klasse RessourceMarket}}\\
  Der Ressourcenmarkt setzt sich aus den drei unterschiedlichen Rohstoffen
  zusammen, die in unserem SPiel vorhanden sind. Die Preise f�r die Rohstoffe
  sind in dem Programm als Konstanten festgesetzt und �ndern sich nicht mit der
  Zeit.
  \item{\textbf{Klasse EnergyExchange}}\\
  Die Stromb�rse im Spiel hat einen schwankenden Preis. Dieser reguliert sich
  aus Angebot und Nachfrage. Jedes mal wenn ein Spieler einen Stromhandel
  abschlie�t, schickt der Client eine kleine Nachricht �ber die h�he des Handels
  an den Server, der daraufhin den neuen Preis f�r die n�chste Runde berechnet.
  \item{\textbf{Klasse Region und Klasse Coords}}\\
  Die Klasse Region ist daf�r gedacht, dass sie mit ihren Koordinaten, die in
  der Klasse Coords gespeichert sind, eindeutig ermittelt werden kann. Die
  Region ist nich daf�r gedacht, dass sie f�r sich instanziiert wird, sondern
  mit Hilfe von Vererbung entweder als Ressourcenregion oder Stadtregion benutzt
  wird.
  \item{\textbf{Klasse CityRegion und Klasse Contract}}\\
  Die Stadtregion enth�lt alle Informationen �ber die Einwohner der Stadt und
  die Vertr�ge mit den verschiedenen Energielieferanten. Die Stadt wird mit
  einer Einwohnerzahl und verschiedenen Kennziffern zum Spielstart generiert,
  w�hrend die Kundenzahlen, die in den Contracts gespeichert sind, sich jede
  Runde durch Preis�nderungen der Spieler �ndern k�nnen. Jede Stadt hat f�r
  jedes Unternehmen, dass in ihr Kunden besitzt, ein seperates Contract-Objekt.
  \item{\textbf{Klasse RessourceRegion}}\\ 
  Die RessourceRegion wird mit einem bestimmten RessourceType (siehe
  Enumeration RessouurceType) und gegebenenfalls einer Menge des Rohstoffes zum
  Spielstart generiert. F�r erneuerbare Energien, wie z.B. Wasser entf�llt
  nat�rlich die Rohstoffmenge. W�hrend des Spiels k�nnen Spieler auf die
  Regionen bieten, um sie bebauen zu k�nnen. Diese Gebote werden an den Server
  gesendet, der dann den Zuschlag an ein Unternehmen gibt, der danach zum
  Besitzer der Region wird. Zudem meldet der Besitzer der Region dem Server,
  sobald er ein Geb�ude in einer Region fertiggestellt hat. Dieser schreibt dies
  dann in das Region-Objekt, sodass dies auch f�r die anderen Spieler sichtbar
  wird.
  
\end{itemize}


\subsection{Geb�ude}



PowerstatioNrelation stellt die verbindung zwischen einem Kraftwerk und einer
Stadt mit Vertrag da \ref{Opt} 

\section{Finanzen}

\begin{figure}[H]
\centering
\includegraphics[width=1.0\textwidth]{se-wa-jpg/finances}
\caption{Klassendiagramm zum Bereich der Finanzen}
\label{Finanzen}
\end{figure}
Die Finanzen sollen im Wesentlichen alle finanziell relevanten Aspekte des
Unternehmens abbilden. Hierzu gibt es das Department Finances, welches die
Bilanz, die Gewinn- und Verlustrechnung sowie alle vom Unternehmen aufgenommenen
Kredite verwaltet.

Im Einzelnen werden die relevantesten Methoden und ihre Aufgabe vorgestellt:\\
\textbf{Klasse Finances}
\begin{itemize}
  \item \textbf{addCredit(CreditType creditType):void}\\
  Mit dieser Methode lassen sich beliebig viele Kredite dem Unternehmen
  hinzuf�gen, wobei jeder Kredit als neues Objekt in die ArrayList ``credits''
  der Klasse Finances aufgenommen wird und der Kassenbestand des Unternehmens um die
  Darlehensh�he erh�ht wird. 
  \item \textbf{isCreditWorthyFor(CreditType creditType):boolean}
  Diese Methode gibt zur�ck, ob der vom Spiel vorgesehene maximale
  Verschuldungsgrad bei Aufnahme eines Kredits des Typs creditType �berschritten
  w�rde. Damit kann im UI eine �berpr�fung implementiert werden, die eine zu
  hohe Verschuldung eines Spielers verhindern kann.
  \item \textbf{isInsolvent():boolean}
  Gibt diese Methode true zur�ck, hat der Spieler eine zu hohe Verschuldung
  (welche durch Verluste und damit durch eine Reduktion des Eigenkapitals
  entstehen kann) und verf�gt gleichzeitig �ber einen negativen Kassenbestand.
  Ist das der Fall, hat der Spieler das Spiel verloren.
  \item \textbf{nextRound() und nextYear()}
  werden automatisch aufgerufen und f�hren die Kreditzahlungen durch.
\end{itemize}
\textbf{Klasse Credit}
\begin{itemize}
  \item \textbf{payQuarter()}
  wird jede Runde aufgerufen und f�hrt sowohl die Tilgung als auch die
  Zinszahlung durch. Bei abgezahlten Krediten betragen beide Werte 0, der
  Einfachheit halber (und der Tatsache, dass die Anzahl Kredite stark begrenzt
  ist) werden abbezahlte Kredite jedoch nicht aus der ArrayList credits der
  Klasse Finances entfernt.
\end{itemize}
\textbf{Klasse Balance}
\begin{itemize}
  \item \textbf{recalcBalance()}
  berechnet jedes Jahr die Werte der Grundst�cke und Geb�ude (AV), des Inventars
  und der Kasse (UV), des Eigenkapitals (EK) und des Fremdkapitals (FK).
\end{itemize}
\textbf{Klasse ProfitAndLoss}
\begin{itemize}
  \item \textbf{addXYZ(double amount)}
  da nicht alle Aufwendungen und Ertr�ge zum Ende jeden Jahres berechnet werden
  k�nnen, addieren diese Methoden die entsprechenden Werte f�r die jeweiligen
  A/E und speichern sie in den Attributen \textbf{nextXYZ}.
  \item \textbf{nextYear()}
  In dieser Methode wird jedes Jahr die Gewinn- und Verlustrechnung neu
  berechnet. Dazu werden die Werte der Variablen nextXYZ in die Variablen XYZ
  �berschrieben und dann auf 0 gesetzt. Alle anderen Aufwendungen und Ertr�ge
  werden nun berechnet. Anschlie�end wird der Gewinn bestimmt, von dem eine
  Steuer in H�he von 30\% abgezogen wird, sofern ein Gewinn vorhanden ist.
  Anschlie�end wird der Kassenbestand des Unternehmens um die H�he der
  anfallenden Steuern reduziert.
\end{itemize}

\section{Warenlager}

\begin{figure}[H]
\centering
\includegraphics[width=1.0\textwidth]{se-wa-jpg/Warehouse}
\caption{Klassendiagramm zum Warenlager}
\label{Warenlager}
\end{figure}
Das Warehouse beinhaltet die Verwaltung aller Rohstoffe eines Unternehmens. Es
bildet die Lagerung durch eine Komposition mit der Klasse Ware ab, es enth�lt
Methoden zur Erh�hung und Reduzierung des Bestands sowie zum Kauf und Verkauf
von Resourcen.

\textbf{Klasse Warehouse}
\begin{itemize}
  \item \textbf{buyWare(\ldots) / sellWare(\ldots)}
  dient zum Kauf und Verkauf von Waren. Hierbei muss nicht nur der Kassenbestand
  des Unternehmens aktualisiert werden, es entstehen gleichzeitig auch
  Aufwendungen / Ertr�ge.
  \item \textbf{getStoredValue()}
  gibt den Wert der gelagerten Rohstoffe zur�ck. Dies wird zur Berechnung der
  Bilanz ben�tigt.
  \item \textbf{nextRound()}
  die Lagerkosten werden um einen von der Menge der gelagerten G�ter abh�ngigen
  Wert erh�ht.
  \item \textbf{nextYear()}
  Die Bestandsver�nderungen im Vergleich zum letzten Jahr werden berechnet. Dies
  ist f�r die Gewinn- und Verlustrechnung erforderlich, da hier Aufwendungen
  bzw. Ertr�ge entstehen.
\end{itemize}

% Wordcount: 490
%Marco
\section{Design}
Um den Spielern des Planspiels auch ein echtes Spielerlebnis zu bieten wurde die
Entscheidung getroffen eine grafische Oberfl�che zu schaffen, welche jegliche Spielinformationen abbildet. 
Auf den folgenden Seiten wird der Entwicklungsprozess, als auch die Ergebnisse
und m�gliche Verbesserungsm�glichkeiten wiedergegeben:

\subsection{Entwicklung} 
Die Entwicklung des Designs spielte eine wichtige Rolle bei uns, welche sich in einzelne Phasen unterteilen lassen kann. 
Die einzelnen Phasen, Schritte und Ideen dahinter werden im folgenden dargelegt.

\subsubsection{Brainstorming \& erste Skizzen}
Nach der Entwicklung des Spielkonzeptes und der Programmierung der ersten
Spielfunktionen wurde damit begonnen die ersten Ideen f�r das User-Interface zu sammeln. 
Dazu wurde eine gemeinsame Brainstorming-Session gestartet und alle m�glichen
Vorschl�ge zusammengetragen. Diese Vorschl�ge wurden dann in erste Skizzen
verpackt und m�gliche Anordnungen und Elemente diskutiert. 
Da die Skizzen jedoch sehr einfach, un�bersichtlich \& ungenau waren wurden
daraus, mit Hilfe von Grafikprogramme, Mockups erstellt, um die Ideen
realit�tsgenau darzustellen und  wirken zu lassen und somit sowohl St�rken als
auch Schw�chen der Ideen aufzudecken.

\begin{figure}[H]
\centering
\centering
\includegraphics[width=0.8\textwidth]{se-wa-jpg/gui_skizze}
\caption{Eine der ersten Skizzen aus dem Brainstorming}
\label{Eine der ersten Skizzen aus dem Brainstorming}
\end{figure}

\subsubsection{Mockup \& Ideen}
Im folgenden werden die einzelnen Mockup-Grafiken dargestellt und kurz
erl�utert:

\paragraph{Elemente}
Um den Spieler nicht zu verwirren wird auf statische Oberfl�chenelemente
gesetzt, welche sich in jedem Bildschirm an der selben Stellen befinden. Daf�r
wurde der Bildschirm der Mockups in drei Teile geteilt:
\begin{itemize}
\item Men�leiste
\item Hauptframe
\item Sidebar
\end{itemize}

Die \textbf{Men�leiste} befindet sich am oberen Bildschirmrand und stellt dem
Spieler die Navigation auf die unterschiedlichen Seiten zur Verf�gung.

Im \textbf{Hauptfrage} werden die Hauptinformationen des Spiels dargestellt und
ist f�r die Hauptinteraktionen verantwortlich.

Die \textbf{Sidebar} dient dabei als Erweiterung und bietet zus�tzliche
Informationen, aber auch Interaktionsm�glichkeiten an.

\paragraph{Unternehmens�bersicht}
Die Idee hinter der Unternehmens�bersicht ist, dass der Spieler einen �berblick
�ber alle wichtigen Informationen seines Unternehmens bekommt. Dazu z�hlt der
Kontostand, Einnahmen, Ausgaben, Anzahl der Rohstoffe, Stromproduktion,
Stromverbrauch, Strombilanz und die Anzahl der im Besitz befindenen Kraftwerke & Mienen. Des Weiteren soll der Nutzer auf dieser Seite auch �ber Aktionen der vorangegangen Spielrunde informiert werden, um keine wichtigen Interkationen & Ereignisse zu verpassen.
\begin{figure}[H]
\centering
\centering
\includegraphics[width=0.9\textwidth]{se-wa-jpg/gui_unternehmen}
\caption{Unternehmens�bersicht}
\label{Unternehmens�bersicht}
\end{figure}

\paragraph{Karte mit ausgew�hltem Ressourcenfeld}
\begin{figure}[H]
\centering
\centering
\includegraphics[width=0.9\textwidth]{se-wa-jpg/gui_karte01}
\caption{Karte mit ausgew�hltem Ressourcenfeld}
\label{Karte mit ausgew�hltem Ressourcenfeld}
\end{figure}

\paragraph{Karte mit ausgew�hlter Stadt}
\begin{figure}[H]
\centering
\centering
\includegraphics[width=0.9\textwidth]{se-wa-jpg/gui_karte02}
\caption{Karte mit ausgew�hlter Stadt}
\label{Karte mit ausgew�hlter Stadt}
\end{figure}

\paragraph{Finanzen}
\begin{figure}[H]
\centering
\centering
\includegraphics[width=0.9\textwidth]{se-wa-jpg/gui_finanzen}
\caption{Finanzen}
\label{Finanzen}
\end{figure}

\paragraph{Markt}
\begin{figure}[H]
\centering
\centering
\includegraphics[width=0.9\textwidth]{se-wa-jpg/gui_markt}
\caption{Markt}
\label{Markt}
\end{figure}

\subsection{Auswertung}
\subsubsection{Ergebnis}

\subsubsection{Vergleich Mockup mit Ergebnis}
\subsubsection{Verbesserungsm�glichkeiten}
%Olli
\section{Graphische Benutzeroberfl�che}\label{olli:gui}
In dieser Sektion wird der Aufbau der graphischen Benutzeroberfl�che (GUI)
erl�utert, die mit Java Swing entwickelt wurde. Zun�chst wird das GUI durch die
Singleton Klasse \textbf{Controller} verwaltet und durch diese wird
sichergestellt, dass von dem GUI die aktuellen Informationen angezeigt werden.
Des Weiteren bietet es den GUI Klassen den Zugriff auf das Datenmodell. Die GUI
Klassen liegen im Paket \textbf{de.client.gui}. 
%In der �berwiegenden Mehrheit
%sind einzelne Komponenten durch lazy initialization

\subsection{Globale Bestandteile}
Allgemein werden die ben�tigten JPanels immer in ein Objekt der Klasse
\textbf{Frame} eingef�gt --- dies ist das Fenster, das f�r das Spiel verwendet
wird.
\begin{itemize}
  \item Ein \textbf{PanelGlobalLeft} befindet sich immer auf der linken Seite
  des Fensters, in dem die Liste von Spielern mit ihrem Status und der Chat 
  enthalten sind. Diese beiden Einzelteile werden von dem Controller
  aktualisiert, wenn der Client eine Nachricht vom Typ  
  \texttt{PLAYER\_READY\_CHANGE} bzw. \texttt{CHAT\_BROADCAST} erh�lt.
  \item Das Men� wird durch ein \textbf{PanelMenu} dargestellt, das die Buttons
  und ein \textbf{PanelMenuInformation} enth�lt. \textbf{PanelMenuInformation}
  besteht aus einem JLabel mit einem Text, der HTML Code enth�lt --- dadurch
  entsteht die Tabellenstruktur. 
  \item \textbf{PanelAbstractContent} wird bei allen Tabs au�er dem Markt
  verwendet: In dieser abstrakten Klasse werden zwei JPanels definiert, dessen
  Layout mithilfe eines \textbf{GridBagLayout} in der Methode
  \textbf{initLayout()} festgelegt wird. Durch die Festlegung eines linken und
  eines rechten Panels soll ein einheitliches Layout in allen Tabs entstehen ---
  des Weiteren wird hier bereits die Hintergrundfarbe und die R�nder gesetzt.
\end{itemize}

Wichtig ist au�erdem die Klasse \textbf{Look}, in dem statisch die verschiedenen
Hintergrundfarben und die R�nder der hexagonf�rmigen Buttons in der Karte
definiert sind. Des Weiteren wurde die Klasse \textbf{Strings} erstellt, die
mehrere statische Methoden bereit stellt, die Daten f�r die Anzeige im GUI vorbereiten
soll --- darunter zum Beispiel die Funktion \textbf{fD(double number)}, die
Zahlen, besonders vom Typ \textbf{double}, lesbarer machen soll. Au�erdem
besitzt sie Attribute wie \textbf{ENERGY\_UNIT}, damit diese an einer Stelle
definiert sind und auch einheitlich ver�ndert werden k�nnen.

Nach dem Mockup wurden vier unterschiedliche Tabs entwickelt, die dem Spieler
unterschiedliche Einblicke in sein Unternehmen geben sollen:

\subsection{Unternehmensplanung}
Die Unternehmensplanung ist �hnlich wie die meisten anderen Tabs aus zwei
Panels aufgebaut und wird durch ein \textbf{PanelCompany} dargestellt, das von 
\textbf{PanelAbstractContent} erbt. 

Das linke (im Fenster mittlere) Panel ist ein \textbf{GridLayout} mit zwei
Spalten und zwei Reihen und wird von JLabels gef�llt. Die Tabellenstruktur in
den einzelnen JLabels entsteht �hnlich wie im Men� durch die Nutzung von HTML 
Code. Die rechte Seite enth�lt die Nachrichten, die dem Spieler zum Beginn der 
Runde angezeigt werden.

\subsection{Karte}\label{GUIKarte}
Die Karte ist das Herzst�ck der Benutzeroberfl�che --- hier kann der Spieler die
wichtigsten Entscheidungen wie das Bauen von Kraftwerken und das Schlie�en von
Vertr�gen treffen. Insgesamt ist der Kartentab ein \textbf{PanelMain}, das
ebenfalls von \textbf{PanelAbstractContent} erbt. Zun�chst wird die linke (im
Fenster mittlere) Seite erkl�rt.

Das \textbf{PanelMap} selbst enth�lt nur viele \textbf{HexagonButton}, dessen
Positionen von einem \textbf{HexagonLayout} verwaltet werden. Die Implementation
des HexagonLayout wurde anfangs von (\seCite{}{}{codeguru})  �bernommen. Dieses
besa� am Anfang nur ein rechteckiges Layout --- es wurde anschlie�end angepasst,
damit beispielsweise die H�he und Breite der Buttons korrekt bei �nderung der 
Fenstergr��e mitskaliert werden. Des Weiteren wurde ein hexagonf�rmiges Layout 
implementiert, das unserer Spielidee n�her liegt: Das GUI kann beide 
Kartenarten darstellen.

Jeder \textbf{HexagonButton} erh�lt bei der Initialisierung eine
\textbf{Region}, mit der er seine Hintergrundfarbe, R�nder und das Bild
bestimmten kann. Des Weiteren besitzt er eine Referenz zu einem
\textbf{PanelSubDetails}: Diese wird bei dem ersten Klick auf dem Button
initialisiert und bleibt danach erhalten. Der Inhalt davon wird durch einen
Entscheidungsbaum in \textbf{PanelDetails.setRegionContent} bestimmt.

Durch ``Lazy Initialization'' des \textbf{PanelSubDetails} wird hier auf lange
Sicht Rechenzeit gespart, daf�r wird jedoch mehr Speicherplatz ben�tigt. Das
\textbf{PanelSubDetails} wird dann jeweils in die rechte Seite des
Kartentabs, die durch ein \textbf{PanelDetails} dargestellt wird, eingef�gt.

Erh�lt der Client durch eine \textbf{MAP\_UPDATE} Nachricht ist die Runde
beendet: Die Referenz zum alten \textbf{PanelMap} wird gel�scht und ein komplett
neues Objekt wird angelegt --- deswegen flackert das GUI kurz, sobald eine Runde
beendet ist.

\subsection{Finanzen}
Die finanziellen Informationen der Firma werden durch ein \textbf{PanelFinances}
abgebildet: Auf der linken Seite befindet sich ein
\textbf{PanelFinancesBalance}, das durch viele verschachtelte 
\textbf{GridLayout} konstruiert wird. Hier wird die Bilanz und die GuV des
Unternehmens dargestellt.Daneben besteht die rechte Seite aus
einem JLabel mit HTML Code und einem \textbf{PanelCredits}, in dem
Informationen zu den verschiedenen Kredittypen angezeigt werden.

\subsection{Markt}
Die beiden Seiten des Marktes sind relativ gleich aufgebaut und befinden sich in
einem \textbf{PanelMarket}. Sie besitzen beide ein \textbf{BoxLayout}, damit
Komponenten vertikal untereinander eingef�gt werden. Die Anzeige der einzelnen
Rohstoffe, die hier gekauft und verkauft werden k�nnen, werden durch ein
\textbf{PanelMarketResource} erstellt und basieren auf der Enumeration
\textbf{de.shared.map.region.BuyableResouce}.
%Olli
\section{Unit Tests}\label{olli:unittests}
In diesem Projekt wurde versucht, m�glichst fl�chendeckend und detailiert Unit
Tests zu erstellen. Deswegen haben wir uns f�r zwei Kategorien von Unit Tests
entschieden: Erstens die, die prim�r die Spiellogik testen sollen und zweitens
Szenariotests, die ganze Spielemechaniken und die Kommunikation zwischen
Client und Server testen sollen.

Damit wurde nach dem Eclipse Plugin ``EclEmma'' insgesamt ein Code Coverage von
ca. 80\% erreicht (ohne Betrachtung des Codes in \textbf{de.tests} und des Codes in
\textbf{de.client.gui}, der die graphische Benutzeroberfl�che generiert).

\subsection{Spiellogik}
Unter diese Kategorie fallen alle Tests, die die Funktionalit�ten des Spiels
testen, die keine Client-Server Kommunikation enthalten.

\begin{itemize}
  \item \textbf{de.tests} enth�lt \textbf{AllTests}: eine gro�e TestSuite, die
  alle Modultests durchf�hrt und \textbf{TestManual}, der gleichzeitig den
  Server und die Benutzeroberfl�che startet, falls ein manueller Test gestartet
  werden soll.
  \item \textbf{de.tests.client}: In diesem Paket sind zwei Tests
  enthalten, die die finanziellen Aspekte des Spieles testen soll --- dazu
  geh�ren die folgenden Tests: \textbf{TestInvestmentDepreciation}, der die
  Abschreibung von Grundst�cken und Geb�uden testet und \textbf{TestWarehouse},
  in dem die Verwaltung von Rohstoffen getestet wird. 
  \item \textbf{de.tests.client.optimization}: Hier wird die Optimierung der
  Kraftwerke getestet. Dazu gibt es eine Hilfsklasse
  \textbf{TestObjectFactory}, die f�r die vier Tests die erforderlichen Objekte statisch erzeugt. 
  \item \textbf{de.tests.client.gui}: Tests in diesem Paket sind keine Unit
  Tests im eigentlichen Sinne eines automatischen Durchganges sondern wurden erstellt, um
  das Testen der graphischen Benutzeroberfl�che zu erleichtern. F�r jeden Tab
  der Benutzeroberfl�che gibt es eine Klasse.
  \item \textbf{de.tests.shared.map}: In dem letzten Paket wird durch
  \textbf{GenerateRegions} getestet, ob die Regionen richtig erzeugt werden und
  in \textbf{TestEnergyExchange} werden die Funktionalit�ten der Energieb�rse
  getstet. 
\end{itemize}

\subsection{Szenariotests}
Die zweite Kategorie von Unit Tests sind die Tests in
\textbf{de.tests.clientserver}, die die Kommunikation zwischen den Clients und
dem Server testen. Dabei ist zun�chst wichtig, dass die Klasse \textbf{Server}
das Singleton Pattern implementiert und deswegen zwischen den einzelnen Tests
mit der Methode \textbf{reset()} zur�ckgesetzt werden muss --- hier wird das
\textbf{ServerSocket} geschlossen und wieder neu er�ffnet.

Des Weiteren erbt jeder Test in diesem Paket von
\textbf{AbstractClientServerTest}, der \textbf{@Before} und \textbf{@After}
Methoden definiert, damit der Server zwischen den Tests neugestartet wird.

Bei einem JUnit Testdurchlauf werden alle Methoden, die als \textbf{@Test}
markiert sind, als neue Thread gestartet und parallel ausgef�hrt. Da dies bei gleichzeitigen
Client-Server Tests zu Schwierigkeiten wegen des \textbf{ServerSocket}  f�hren
w�rde, wurde die Klasse \textbf{ClientServerMonitor} konzipiert. Zun�chst
implementiert diese aus zwei Gr�nden das Singleton Pattern:
Erstens muss ein Monitor f�r die \textbf{Object.wait()} Methode objektorientiert und nicht statisch sein
und zweitens um eine Referenz bei jeder Testklasse zu sparen. Die Methode 
\textbf{startTest()} wird vor dem Start eines Tests aufgerufen --- damit werden
alle anderen Threads, die auch dieses Methode aufrufen, durch
Warten ausgeschlossen. Ist ein Test beendet wird die \textbf{endTest()} Methode
aufgerufen, die dann alle anderen Test Threads wieder aufweckt. Durch die
Nutzungs eines Monitors kann hier also sichergestellt werden, dass die
Client-Server Tests trotz parallelem Starten hintereinander ausgef�hrt werden.


Es wird in der \textbf{AbstractClientServerTest} Klasse Folgendes
aufgerufen: 
\begin{itemize}
  \item \textbf{@Before}: \textbf{ClientServerMonitor.getInstance().startTest()}
  \item \textbf{@After}: \textbf{ClientServerMonitor.getInstance().endTest()}
\end{itemize}
Zuletzt gibt es die Klasse \textbf{TestUtils}: Da die Clients und der Server
nicht in der Hauptthread ausgef�hrt werden muss dieser blockiert werden, wenn
Nachrichten verschickt werden. Dazu wurde die statische Methode
\textbf{TestUtils.block()} geschrieben, die den Hauptthread f�r eine
bestimmte Dauer durch eine while-Schleife blockiert. Dadurch kann die Dauer der
Unit Tests allerdings auch sehr in die L�nge gezogen werden. \textit{Sollten die
Client-Server Unit Tests einen Fehler anzeigen k�nnte es sein, dass die
Blockierzeit nicht hoch genug ist. Dies ist von Computer zu Computer unterschiedlich!}

%Patrick
\section{Balancing der Betriebswirtschaft}\label{patrick:balancing}
\subsection{Umsetzung betriebswirtschaftlicher Besonderheiten}
Die Besonderheiten, die wir bei der Analyse der betriebswirtschaftlichen
Grundlagen erarbeitet haben, wurden in unserem Planspiel verwirklicht, in dem
die Stromproduktion verschiedener Kraftwerke unterschiedlich geregelt werden kann.

Der Betrieb von Kern- und Kohlekraftwerken wiederum sorgt f�r Unzufriedenheit in
der Bev�lkerung und senkt somit die Beliebtheit des Betreibers. Diese
Funktionalit�t wurde auch in das Spiel integriert und hat Auswirkungen auf die
Zahl der Vertr�ge, die in einer Stadt abgeschlossen werden.
Die Stadt ist neben den ``Baufeldern'' das zentrale Medium um einen finanziellen
Umsatz in dem Spiel zu generieren. Durch das Anbieten einer verf�gbaren
Kapazit�t zu einem festgelegten Preis, wird ein Vertrag mit der Stadt und somit
einer gewissen Anzahl an Kunden erstellt. Je nach Anbietern bei dieser Stadt und
je nach vorgeschlagenem Preis, hat das Unternehmen eine gewisse Anzahl an
Kunden.

Die weitere M�glichkeit, �ber die im Planspiel Geld verdient werden kann ist,
den Strom �ber die Stromb�rse zu verkaufen. Allerdings ist diese so
programmiert, dass sie sich auch wie an einer ``echten'' B�rse verh�lt, das hei�t
je mehr Angebot vorhanden ist desto geringer ist der Preis.
Eine dritte M�glichkeit Geld zu verdienen, welche allerdings der Spielidee
entgegensteht, ist es, nach dem Bau von Minen die dort produzierten Rohstoffe an
dem Markt zu verkaufen.

\newpage
\subsection{Festlegung betriebswirtschaftlicher Werte im Planspiel}
Im Folgenden ist eine Kosten�bersicht, die neben den laufenden Kosten und den
Baukosten auch die Bauzeit und die Leistung der einzelnen Kraftwerke beinhaltet.

%Tabelle 1
\begin{figure}[H]
\centering
\includegraphics[width=1\textwidth]{se-wa-jpg/tabelle-1-patrick}
\caption{�bersicht der Kraftwerkskosten und -bauzeit}
\label{�bersicht Kraftwerkseigenschaften}
\end{figure}

Die obige Tabelle zeigt die im Spiel implementierten, von Anfang an festgelegten
Daten. Dabei handelt sich um eine reine �bersicht der Baukosten und -zeit und
die laufenden Kosten. Ein Kernkraftwerk kostet 3,1 Milliarden Euro und es dauert
10 Quartale, eins zu bauen. Damit ist dies das teuerste Kraftwerk in der
Anschaffung, allerdings sind die laufenden Kosten nicht so hoch und durch den
gro�en Leistungsunterschied zu den anderen Kraftwerken ist es m�glich, einen
hohen Gewinn erreichen. 
Im Gegensatz dazu kann eine Photovoltaikanlage innerhalb von einem Quartal
gebaut werden und kostet 45 Millionen Euro, wodurch dies das g�nstigste
Kraftwerk in der Anschaffung ist. Allerdings betr�gt die Leistung nur ein
ein f�nfzigstel von dem eines Kernkraftwerks.

Alle Werte wurden so festgelegt, dass die Bauzeit, die Leistung, die Baukosten
sowie die laufenden Kosten in Relation zueinander stehen.
(Willst du hier nicht wenigstens jedes Kraftwerk mal kurz konkret erw�hnen mit seinen Daten?)

Abgesehen vom Bau der Kraftwerke besteht auch noch die M�glichkeit, auf den
Kohle-, Gas- und Kernkraftwerkfeldern Minen zu bauen. Diese Felder produzieren
Kohle, Gas beziehungsweise Uran. Diese Rohstoffe werden ben�tigt, um die Kraftwerke zu 
betreiben, wenn diese nicht auf den jeweiligen Feldern abgebaut werden, muss der Spieler
Rohstoffe auf dem Handelsplatz holen.
Die Minen haben in unserem Spiel, wie auch in der Realit�t eine gewisse zuf�llig generierte Kapazit�t, 
daher sollte der Spieler die Kapazit�t der Mine im Auge behalten und gegenebenenfalls fr�hzeitig
auf andere Alternativen setzen. Welche sein k�nnten: Neue Felder zu erschlie�en um dort
Minen zu bauen oder alternativ konstant Rohstoffe vom Markt zu zukaufen und als letzte Methode
einen Umstieg von den Fossilen Brennstoffen auf die erneuerbaren Energien einzuleiten.

%Tabelle 2
\begin{figure}[H]
\centering
\includegraphics[width=1\textwidth]{se-wa-jpg/tabelle-2-patrick}
\caption{�bersicht der Minenkosten und -bauzeit}
\label{�bersicht Mineneigenschaften}
\end{figure}

Auch die Minen m�ssen erst gebaut werden, was eine gewisse Zeit in Anspruch
nimmt und wobei auch Kosten entstehen. Hierbei ist zum Beispiel ein Augenmerk
auf die Gas-Mine zu legen, da hier die Bauzeit l�nger ist, als der eigentliche 
Bau eines Kraftwerks. Somit muss entweder erst die Mine gebaut werden, bevor das
Kraftwerk in Auftrag gegeben wird oder die ben�tigten Rohstoffe werden f�r die
ersten zwei Quartale von dem Rohstoffmarkt bezogen. Durch den Abbau der
jeweiligen Rohstoffe sind auch bei einer Mine laufende Kosten zu entrichten.

Die letzten beiden Spalten der Grafik beschreiben sowohl die Menge an
Rohstoffen, die ein Kraftwerk pro Quartal ben�tigt, als auch die Kosten f�r den
Bezug der Rohstoffe vom Rohstoffmarkt. Diese Kosten sind nicht variabel sondern
bleiben stetig gleich.

Die Festlegung des Startkapitals erfolgte erst nach der Bestimmung aller anderen
Werte, da erst danach ein sinnvoller Startwert, aufbauend auf diesen Werten,
ermittelt werden konnte. Der Wert von 650 Millionen Euro war das Ergebnis
unserer Ermittlungen. Mit diesem Startkapital kann sowohl die zweitgr��te
Kraftwerksart gebaut werden, als auch einige kleinere, sodass jeder Spieler die
M�glichkeit hat, sich je nach Interesse im Markt auszurichten.

Falls ein Spieler eine gr��ere Investition t�tigen m�chte oder Probleme mit
liquiden Mitteln hat, besteht die M�glichkeit Kredite aufzunehmen. Diese sind in
drei unterschiedliche Kreditgr��en aufgegliedert:

\begin{itemize}
  \item 100 Millionen
  \item 500 Millionen
  \item 1 Milliarde
\end{itemize}

Dem Spieler steht die M�glichkeit zur Verf�gung so viele Kredite aufzunehmen wie
er m�chte, allerdings darf die gesamte Kreditsumme dabei nicht mehr als das
Zweifache des Eigenkapitals betragen.
Die Kredite sind allesamt mit einem Prozentsatz von 5 \% verzinst. Die Kredite
sind des Weiteren allesamt Tilgungskredite. Diese sind statisch und damit auch
verh�ltnism��ig einfach zu implementieren. Das aufgenommene Darlehen wird �ber
eine feste Laufzeit und einer gleichbleibenden Tilgung zur�ckgezahlt.




\subsection{Ums�tze}

%Tabelle 3
\begin{figure}[H]
\centering
\includegraphics[width=1.02\textwidth]{se-wa-jpg/Patrick-tabelle3}
\caption{�bersicht der Kraftwerksums�tze und -gewinne}
\label{�bersicht Umst�ze und Gewinne}
\end{figure}

Die Grafik zeigt die Ums�tze der einzelnen Kraftwerksarten und eine �bersicht
der gesamten Kosten und den daraus resultierenden Gewinn. F�r die drei
Kraftwerksarten, bei denen es Mineralien gibt wurde eine extra �bersicht
erstellt, wie sich die Kosten und Gewinne entwickeln w�rden, mit beziehungsweise ohne Mine.
Auf den ersten Blick macht es den Eindruck, dass die Gewinne sehr stark
auseinander liegen, allerdings ist hierbei drauf zu achten, dass die
unterschiedlichen Baukosten und Bauzeiten, in dieser �bersicht nicht mit einbezogen sind.

Die Photovoltaikanlagen haben die geringste Marge, allerdings sind hierf�r auch keine Kosten
f�r die Rohstoffe mit einzubeziehen. Hierbei ist Wind mit einer gesamt Gewinnmarge von etwas mehr als
46 Millionen das effektivste Kraftwerk unter den Erneuerbaren.\\ 
Unter den Kraftwerken die mit fossilen Brennstoff funktionieren, ist bei Betrieb mit
als auch ohne Mine das Kernkraftwerk das mit dem h�chsten Gewinn, mit weitem Abstand zu Kohle und Gas.
Der Gewinn eines Kohlekraftwerks ist h�her, gegen�ber einem Gaskraftwerk, l�sst sich wiederum nicht so leicht regeln.
  


\newpage
\chapter{Fazit}
%Matthias
\section{Kritische Betrachtung Matthias}
%Matthias
\section{Ausblick}
Ein wichtiges Ziel dieser Arbeit war das eigene Organisieren und Erfahren eines
kompletten Projekts des Software Engineering.\\
Der Ablauf mit Planungsphase, Analyse, Design, Entwurf sowie das Arbeiten mit
standardisierten Modellen wie den UML-Notationen haben wir am eigenen Leib
erfahren. Auch die negativen Aspekte wie Fehlplanung und Zeitdruck haben wir
erlebt, ohne die sich ein Projekt vermutlich gar nicht realisieren l�sst.\\
Unter Anwendung von statischen und dynamischen Modellen haben wir als Team an
einem St�ck Software entwickelt, was sich trotz (oder gerade wegen)
Hilfsprogrammen wie Git nicht immer als einfach erwiesen hat.
Vor allem eine sinnvolle Aufgabenteilung fiel zu Beginn schwer, Absprachen
wurden teilweise gar nicht erst getroffen, Missverst�ndnisse f�hrten zu
langwierigen Diskussionen.\\
Doch nachdem sich alle �ber die grundlegenden Ideen einig waren und der
Zeitdruck die Lust auf eben jene Diskussionen schwinden lie�, war ein klarer
Anstieg der Produktivit�t zu bemerken und das Gef�hl, zusammen und nicht jeder
f�r sich an dem Coding zu arbeiten, stellte sich langsam ein, auch wenn
sich immer wieder technische Schwierigkeiten wie Konflikte in Git oder
plattformabh�ngige Fehler bei der Darstellung des GUIs einstellten.

Schlie�lich haben wir es aber geschafft, zur Abgabe ein lauff�higes
Unternehmensplanspiel mit vielen Features und einem funktionierenden User
Interface abzugeben.\\
Gerade im Bereich des Projektmanagements aber auch in technischen Bereichen wie
dem kollaborativen Programmieren an einer Software haben wir viele Erfahrungen
gesammelt, die man in einer gew�hnlichen Vorlesung so nicht h�tte erfahren
k�nnen.

Abschlie�end l�sst sich definitiv behaupten, dass wir viel gelernt haben und
auch der Spa� zwischendurch nicht zu kurz kam.

% Wordcount: 250



% Anhang der Arbeit
% 
%

% Der Anhang sollte auf einer neuen Seite beginnen; daher wird der Seitenvorschub bei neuen Kapiteln 
% wieder angeschaltet; Achtung: die Verwendung von newpage erzeugt eine Kopfzeile, was dann nicht zu dem 
% Gesamtlayout des Dokuments passt
%
%
\seChaptersNewpage
\seAppendix{}



%
%  Erzeugung eines Glossars
%
% Achtung: Das Glossar wird nur ausgegeben, wenn mindestens ein Eintrag in der Arbeit 
%                definiert wurde
%
%

% Die folgenden Kapitel beginnen jeweils auf einer neuen Seite
%
%
\seChaptersNewpage{}
\newpage
\sePrintGlossary{}


%
% Literaturverzeichnisses
%
%\newpage
\sePrintBibliography{}

%\input{\seWaPathText/se-test-literaturverzeichnis}


%
% Festlegung des grundlegenden Formatierungsstils des Literaturverzeichnis
%
\bibliographystyle{jurabib}

% Eigentliche Ausgabe der in der Arbeit verwendeten Quellen
%
%
% Angabe der bib-Dateien, in denen die Quellen beschrieben sind;
% die Angabe geht davon aus, dass eine wa.bib-Datei in demselben 
% Verzeichnis liegt, wie se-ba-vorlage.tex
%

% 2012-02-06
%
% Umbenennung von Literatur- in Quellenverzeichnis
% 
%\renewcommand*{\bibname}{Quellenverzeichnis}
\seBibliography{literatur}


%
% Erzeugung der ehrenw\"ortlichen Erkl\"arung
%
% Der optionale Parameter kann verwendet werden, um f\"ur das Thema der Arbeit eine 
% andere Formatierung vorzunehmen; das sollte in der Regel nicht erforderlich sein;
% ausserdem besteht die Gefahr inkonsistenter Titel auf dem Titelblatt und in der 
% ehrenw\"ortlichen Erkl\"arung
%
%\seEhrenwoertlicheErklaerung{} % dieses Kommando sollte standardm\"assig verwendet werden
\seEhrenwoertlicheErklaerung[\LaTeX-Vorlage zur Anfertigung einer Seminararbeit (Version \version{})]


\end{document}











