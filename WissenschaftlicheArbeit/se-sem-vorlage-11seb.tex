% Konfigurationsdatei f\"ur die Pfaddefinitionen einlesen
\input{se-wa-pfade}
%
%
% Festlegung der Sprache: 
\newcommand{\seWaSprache}{deutsch}
%\newcommand{\seWaSprache}{englisch}

%
% Einlesen der .sty-Dateien
%
%  se-wa-input-styles-v095.tex
%
%  Joerg Baumgart 01.08.2011
%
%  Zusammenfassung und Konfiguration wichtiger Styles f\"ur die 
%  Erzeugung von Seminar-, Projekt- und Bachelorarbeiten
%
%  2012-03-12: auf Version 0.94 umgestellt
%
%
% 2012-12-13: auf Version 0.95 umgestellt
%                     Sprachoptionen englisch/deutsch zusammengef\"uhrt
%                     bchart.sty hinzugenommen
%                 
%


\documentclass[12pt,BCOR=10mm,headinclude=on,footinclude=off,bibliography=totoc]{scrreprt}
\usepackage[T1]{fontenc}
\usepackage[latin1]{inputenc}
\usepackage{ifthen}
% 2012-12-13
\ifthenelse{\equal{\seWaSprache}{deutsch}}{% Deutsche Einstellungen
\usepackage[ngerman]{babel}% 
}{% Englische Einstellungen
\usepackage[english]{babel}% 
}

\usepackage{lmodern}

\usepackage{tikz} % Graphikpaket, das zu pdfLaTeX kompatibel ist
\usepackage{xkeyval} % Definition von Kommandos mit mehreren optionalen Argumenten
\usepackage{listings} % Formatierung von Programmlistings
\usepackage{graphicx} % Einbinden von Graphiken
\usepackage{color}
\usepackage{\seWaPathSty/slashbox} % Diagonalen in Tabellenfeldern
\usepackage{framed} % Erzeugung schwarzer Linien am linken Rand zur Hervorhebung von Textteilen
\usepackage{caption} % Korrektes Setzen einer mehrzeiligen float-Unterschrift bei neu definierten float-Umgebungen
\usepackage{floatrow}
% 2012-12-13
\usepackage{\seWaPathSty/bchart} % Kommandos zur Erzeugung von Balkendiagrammen


% Es wird jeweils die sty-Datei importiert und entsprechende Konfigurationseinstellungen werden vorgenommen

\usepackage{\seWaPathSty/se-jb-scrpage2} % Formatierung der Kopf- und Fu{\ss}zeilen
\usepackage{\seWaPathSty/se-jb-footmisc}    % Fussnoten besser formatieren

\usepackage{\seWaPathSty/se-jb-glossaries-v094} % Abk\"urzungsverzeichnis, Symbolverzeichnis, Glossar
   
\usepackage{\seWaPathSty/se-jb-floatrow}    % Definition und Konfiguration von float-Umgebungen (figure, table, die neue programm-Umgebung)
% Achtung: se-jb-varioref muss nach se-jb-floatrow importiert werden; 
% andernfalls ist der counter programm f\"ur die labelformat-Anweisung noch nicht definiert   
\usepackage{\seWaPathSty/se-jb-varioref}   % Definition von Querverweisen
\usepackage{\seWaPathSty/se-jb-chngcntr}   % Kapitelweise oder globale Nummerierung von Abbildungen etc.
   
\usepackage{\seWaPathSty/se-jb-listen} % Definition neuer, besser formatierter Listen
\usepackage{\seWaPathSty/se-jb-kommandos-v095} % neue Kommandos f\"ur Seminar-, Projekt- und Bachelorarbeiten
% 2012-12-13
\ifthenelse{\equal{\seWaSprache}{englisch}}{\usepackage{\seWaPathSty/se-jb-kommandos-englisch}}{}


%
% Individuelle Konfiguration des Dokumentes
%
\input{\seWaPathText/wa-konfiguration}

%
% Definition von Abk\"urzungen, Symbolen und eventuell Glossareintr\"agen
%
\input{\seWaPathText/wa-abkuerzungen} 

\seIstSeminararbeit{}

\newcommand{\version}{0.953}

% 
% Diese Redefinition ist nur f\"ur den Anhang der  
% Vorlage (Hinweise zur Installation und \"Ubersetzung)
% notwendig; f\"ur Ihre Seminar-/Projekt-/Bachelorarbeit spielt sie keine Rolle
%
\renewcommand{\seVorlage}{\jobname}

% Verwendung kleinerer Schriftgroessen f�r die Ueberschriften; sinnvoll bei kurzen Texten.
%
%
\KOMAoption{headings}{small}

% Fuer einen Kapitelanfang wird kein zusaetzlicher vertikaler Abstand erzeugt
%
%
\seNoChapterSkip[-12.25mm]{}

\begin{document}

% Erzeugung des Titelblatts
%
%
%
\seTitelblattSeminararbeit[
%hilfslinien=ja,
%dhbwlogoSkalierung=0.5,
%dhbwlogoDeltaX=2.4,
%dhbwlogoDeltaY=-10,
studiengang=\seWirtschaftsinformatik,
studienrichtung=\seSoftwareEngineering,
thema=Fallstudie,
verfasser={J�rn H�bner, Patrick Knerr, Marco Riege, Matthias Liedtke, Oliver
Frendo},
%verfasserin= Melanie Musterfrau,
kurs=WWI\,12\,SE\,B,
firma=SAP,
% Da im Text ein Komma enthalten ist, muss der Text eingeklammert werden
%studiengangsleiterin=,
studiengangsleiter=Prof. Dr. Herr Holey,
modul=Umsetzung von Methoden der Wirtschaftsinformatik,
lehrveranstaltung=Fallstudie,
%dozentin=,
dozent=Gregor Tielsch
]


% Erzeugung der englischen Kurzfassung (Abstract); Verfasser, Firma und Thema werden automatisch \"ubernommen
%
% Der optionale Parameter kann verwendet werden, um f\"ur das Thema der Arbeit eine 
% andere Formatierung vorzunehmen; das sollte in der Regel nicht erforderlich sein;
% ausserdem besteht die Gefahr inkonsistenter Titel auf dem Titelblatt und in der 
% Kurzfassung
%
%
% Achtung: Das Kommando erzeugt nur dann eine Ausgabe, wenn \seWaSprache den Wert englisch besitzt
%
%
\seAbstract{} % dieses Kommando sollte standardm\"assig verwendet werden

%\seAbstract[\LaTeX-Vorlage zur Anfertigung \seThemaWaArbeit{} (Version \version{})]



% Erzeugung der Kurzfassung; Verfasser, Firma und Thema werden automatisch \"ubernommen
%
% Der optionale Parameter kann verwendet werden, um f\"ur das Thema der Arbeit eine 
% andere Formatierung vorzunehmen; das sollte in der Regel nicht erforderlich sein;
% ausserdem besteht die Gefahr inkonsistenter Titel auf dem Titelblatt und in der 
% Kurzfassung
%
%\seKurzfassung{} % dieses Kommando sollte standardm\"assig verwendet werden

\seKurzfassung[\LaTeX-Vorlage zur Anfertigung einer Seminararbeit (Version \version{})]


% Beispiel f\"ur ein Kapitel, dass vor dem Einleitungskapitel kommt, z. B. ein Vorwort oder eine Danksagung
%\seKapitelVorEinleitung{Vorwort}
%
%Muss jetzt wirklich nicht sein, aber wenn Sie unbedingt (z. B.) Ihrem Haustier f\"ur die Unterst\"utzung bei 
%der Anfertigung der Projektarbeit danken wollen ...; vgl. auch das Dokument \textsl{Empfehlungen und 
%Hinweise zur Anfertigung der zweiten Projektarbeit}


% 2012-02-06 Inhaltsverzeichnis muss vor den weiteren Verzeichnisses kommen
%
%
% Ausgabe des Inhaltsverzeichnisses
%
%
\seInhaltsverzeichnis[%
einrueckung=ja,
gliederungsebenen=4
]

%
% 
% Wenn die Verzeichnisse (ohne Seitenvorschub) nach dem Inhaltsverzeichnis
% kommen sollen, sind die beiden folgenden Kommandos zu verwenden
%
% Ein neues Kapitel beginnt nicht auf einer neuen Seite
%
%
%\seChaptersWithoutNewpage{}

% Erzeugung eines vertikalen Abstands nach diesem Kapitel
%
%\seChapterEndSkip{}


% Ausgabe der verschiedenen Verzeichnisse
% abk: Abk\"urzungsverzeichnis
% sym: Symbolverzeichnis
% abb: Abbildungsverzeichnis
% tab: Tabellenverzeichnis
% prg: Listingverzeichnis
%
%
% Achtung: Abk\"urzungs- und Symbolverzeichnis werden nur ausgegeben, wenn mindest ein Symbol bzw. 
%                mindestens eine Abk\"urzung in der Arbeit verwendet wurden
%
%
% gliederungsebene:
% -- section: die Verzeichnisse werden einem Kapitel "Verzeichnisse" untergliedert
% -- chapter: die Verzeichnisse sind jeweils eigene Kapitel
% imInhaltsverzeichnis: ja/nein -- Sollen die Verzeichnisse im Inhaltsverzeichnis enthalten sein?
\seVerzeichnisse[gliederungsebene=section,imInhaltsverzeichnis=ja]{abk}{sym}{abb}{tab}{prg}


\seChaptersNewpage{}


% Erstes eigentliches Kapitel der Arbeit; typischerweise das Einleitungskapitel;
% hier muss wieder auf die Nummierung mit arabischen Seitenzahlen umgestellt werden
%
\chapter{Einleitung}\pagenumbering{arabic}

\seChaptersWithoutNewpage{}

Hier kommt jetzt ein ein wenig Text.

\seChapterEndSkip{}


% Erstes Hauptkapitel der Arbeit 
%
%
%
\chapter{Der formale Aufbau einer Seminararbeit}
% Mit markright kann eine verk\"urzte Version der \"Uberschrift f\"ur den Seitenkopf generiert werden
%
%
%\markright{Formaler Aufbau}


% Anhang der Arbeit
% 
%

% Der Anhang sollte auf einer neuen Seite beginnen; daher wird der Seitenvorschub bei neuen Kapiteln 
% wieder angeschaltet; Achtung: die Verwendung von newpage erzeugt eine Kopfzeile, was dann nicht zu dem 
% Gesamtlayout des Dokuments passt
%
%
\seChaptersNewpage
\seAppendix{}

\input{\seWaPathText/se-latex-kommandos}

\input{\seWaPathText/se-englisch}

\input{\seWaPathText/se-hinweise-installation}

\input{\seWaPathText/se-nuetzliche-pakete}

\input{\seWaPathText/se-literaturempfehlungen}

\input{\seWaPathText/se-hinweise-literaturverzeichnis}


%
%  Erzeugung eines Glossars
%
% Achtung: Das Glossar wird nur ausgegeben, wenn mindestens ein Eintrag in der Arbeit 
%                definiert wurde
%
%

% Die folgenden Kapitel beginnen jeweils auf einer neuen Seite
%
%
\seChaptersNewpage{}
\newpage
\sePrintGlossary{}


%
% Literaturverzeichnisses
%
%\newpage
\sePrintBibliography{}

\input{\seWaPathText/se-test-literaturverzeichnis}


%
% Festlegung des grundlegenden Formatierungsstils des Literaturverzeichnis
%
\bibliographystyle{jurabib}

% Eigentliche Ausgabe der in der Arbeit verwendeten Quellen
%
%
% Angabe der bib-Dateien, in denen die Quellen beschrieben sind;
% die Angabe geht davon aus, dass eine wa.bib-Datei in demselben 
% Verzeichnis liegt, wie se-ba-vorlage.tex
%

% 2012-02-06
%
% Umbenennung von Literatur- in Quellenverzeichnis
% 
%\renewcommand*{\bibname}{Quellenverzeichnis}
\seBibliography{wa}


%
% Erzeugung der ehrenw\"ortlichen Erkl\"arung
%
% Der optionale Parameter kann verwendet werden, um f\"ur das Thema der Arbeit eine 
% andere Formatierung vorzunehmen; das sollte in der Regel nicht erforderlich sein;
% ausserdem besteht die Gefahr inkonsistenter Titel auf dem Titelblatt und in der 
% ehrenw\"ortlichen Erkl\"arung
%
%\seEhrenwoertlicheErklaerung{} % dieses Kommando sollte standardm\"assig verwendet werden
\seEhrenwoertlicheErklaerung[\LaTeX-Vorlage zur Anfertigung einer Seminararbeit (Version \version{})]


\end{document}











