%Matthias
\section{Ausblick}
Ein wichtiges Ziel dieser Arbeit war das eigene Organisieren und Erfahren eines
kompletten Projekts des Software Engineering.\\
Der Ablauf mit Planungsphase, Analyse, Design, Entwurf sowie das Arbeiten mit
standardisierten Modellen wie den UML-Notationen haben wir am eigenen Leib
erfahren. Auch die negativen Aspekte wie Fehlplanung und Zeitdruck haben wir
erlebt, ohne die sich ein Projekt vermutlich gar nicht realisieren l�sst.\\
Unter Anwendung von statischen und dynamischen Modellen haben wir als Team an
einem St�ck Software entwickelt, was sich trotz (oder gerade wegen)
Hilfsprogrammen wie Git nicht immer als einfach erwiesen hat.
Vor allem eine sinnvolle Aufgabenteilung fiel zu Beginn schwer, Absprachen
wurden teilweise gar nicht erst getroffen, Missverst�ndnisse f�hrten zu
langwierigen Diskussionen.\\
Doch nachdem sich alle �ber die grundlegenden Ideen einig waren und der
Zeitdruck die Lust auf eben jene Diskussionen schwinden lie�, war ein klarer
Anstieg der Produktivit�t zu bemerken und das Gef�hl, zusammen und nicht jeder
f�r sich an dem Coding zu arbeiten, stellte sich langsam ein, auch wenn
sich immer wieder technische Schwierigkeiten wie Konflikte in Git oder
plattformabh�ngige Fehler bei der Darstellung des GUIs einstellten.

Schlie�lich haben wir es aber geschafft, zur Abgabe ein lauff�higes
Unternehmensplanspiel mit vielen Features und einem funktionierenden User
Interface abzugeben.\\
Gerade im Bereich des Projektmanagements aber auch in technischen Bereichen wie
dem kollaborativen Programmieren an einer Software haben wir viele Erfahrungen
gesammelt, die man in einer gew�hnlichen Vorlesung so nicht h�tte erfahren
k�nnen.

Abschlie�end l�sst sich definitiv behaupten, dass wir viel gelernt haben und
auch der Spa� zwischendurch nicht zu kurz kam.

% Wordcount: 250