%Olli
\section{Unit Tests}\label{olli:unittests}
In diesem Projekt wurde versucht, m�glichst fl�chendeckend und detailiert Unit
Tests zu erstellen. Deswegen haben wir uns f�r zwei Kategorien von Unit Tests
entschieden: Erstens die, die prim�r die Spielelogik testen sollen und zweitens
Szenariotests, die ganze Spielemechaniken und die Kommunikation zwischen
Client und Server testen sollen.

Damit wurde nach dem Eclipse Plugin ``EclEmma'' insgesamt ein Code Coverage von
ca. 80\% erreicht (ohne Betrachtung des Codes in \textbf{de.tests} und des Codes in
\textbf{de.client.gui}, der die graphische Benutzeroberfl�che generiert).

\subsection{Spielelogik}
Unter diese Kategorie fallen alle Tests, die die Funktionalit�ten des Spieles,
die keine Client Server Kommunikation enthalten, testen.

\begin{itemize}
  \item \textbf{de.tests} enth�lt \textbf{AllTests}, eine gro�e TestSuite, die
  alle Modultests durchf�hrt und \textbf{TestManual}, der gleichzeitig den
  Server und die Benutzeroberfl�che startet, falls ein manueller Test gestartet
  werden soll.
  \item \textbf{de.tests.client}: In diesem Paket sind zwei Tests
  enthalten, die die finanziellen Aspekte des Spieles testen soll --- dazu
  geh�ren die folgenden Tests: \textbf{TestInvestmentDepreciation}, der die
  Abschreibung von Grundst�cken und Geb�uden testet und \textbf{TestWarehouse},
  in dem die Verwaltung von Rohstoffen getestet wird. 
  \item \textbf{de.tests.client.optimization}: Hier wird die Optimierung der
  Kraftwerke getestet. Dazu gibt es eine Hilfsklasse \textbf{TestObjectFactory}, die f�r
  die vier Tests die erforderlichen Objekte statisch erzeugt. 
  \item \textbf{de.tests.client.gui}: Tests in diesem Paket sind keine Unit
  Tests im eigentlichen Sinne eines automatischen Durchganges sondern wurden erstellt, um
  das Testen der graphischen Benutzeroberfl�che zu erleichtern. F�r jeden Tab
  der Benutzeroberfl�che gibt es eine Klasse.
  \item \textbf{de.tests.shared.map}: In dem letzten Paket wird durch
  \textbf{GenerateRegions} getestet, ob die Regionen richtig erzeugt werden und
  in \textbf{TestEnergyExchange} werden die Funktionalit�ten der Energieb�rse
  getstet. 
\end{itemize}

\subsection{Szenariotests}
Die zweite Kategorie von Unit Tests sind die Tests in
\textbf{de.tests.clientserver}, die die Kommunikation zwischen den Clients und
dem Server testen. Dabei ist zun�chst wichtig, dass die Klasse \textbf{Server}
das Singleton Pattern implementiert und deswegen zwischen den einzelnen Tests
mit der Methode \textbf{reset()} zur�ckgesetzt werden muss --- hier wird das
\textbf{ServerSocket} geschlossen und wieder neu er�ffnet.

Des Weiteren erbt jeder Test in diesem Paket von
\textbf{AbstractClientServerTest}, der \textbf{@Before} und \textbf{@After}
Methoden definiert, damit der Server zwischen den Tests neugestartet wird.

Bei einem JUnit Testdurchlauf werden alle Methoden, die als \textbf{@Test}
markiert sind, als neue Thread gestartet und parallel ausgef�hrt. Da dies bei gleichzeitigen
Client Server Tests zu Schwierigkeiten wegen des \textbf{ServerSocket}  f�hren
w�rde, wurde die Klasse \textbf{ClientServerMonitor} konzipiert. Zun�chst
implementiert diese aus zwei Gr�nden das Singleton Pattern:
Erstens muss ein Monitor f�r die \textbf{Object.wait()} Methode objektorientiert und nicht statisch sein
und zweitens um eine Referenz bei jeder Testklasse zu sparen. Die Methode 
\textbf{startTest()} wird vor dem Start eines Tests aufgerufen --- damit werden
alle anderen Threads, die auch dieses Methode aufrufen, durch
Warten ausgeschlossen. Ist ein Test beendet wird die \textbf{endTest()} Methode
aufgerufen, die dann alle anderen Test Threads wieder aufweckt. 

Es wird also in der \textbf{AbstractClientServerTest} Klasse Folgendes
aufgerufen: 

\begin{itemize}
  \item \textbf{@Before}: \textbf{ClientServerMonitor.getInstance().startTest()}
  \item \textbf{@After}: \textbf{ClientServerMonitor.getInstance().endTest()}
\end{itemize}