%Olli
\section{Vertr�ge}
Ein wichtiger Teil des Spieles ist f�r den Benutzer das Abschlie�en von
Vertr�gen mit St�dten. Dabei soll zwischen den Spielern eine Konkurrenz
entstehen, um einen Bezug zur Realit�t herzustellen. Dieser
Algorithmus wird jede Runde erneut durchlaufen, damit eine Preiskonkurrenz
zwischen Spielern entsteht --- zudem kann sich dadurch die Anzahl der Kunden
�ndern, ohne dass der Spieler Werte �ndert. Bevor der Algorithmus durchgef�hrt
wird, werden alle bestehenden Vertr�ge mit der Stadt nach ihrem Preis
aufsteigend geordnet. Dies hat zur Auswirkung, dass die Preiswahl des Spielers
ausschlaggebend f�r die Anzahl seiner Kunden ist.

Zun�chst wird die Anzahl der Kunden $y_0$ berechnet, die ein Spieler bei dem
durchschnittlichen Preis bei seiner Bekanntheit und Beliebtheit erhalten w�rde. 
$$y_0 = Bekanntheit * Beliebtheit * Population$$
Anschlie�end wird diese Variable der ``Standardkunden'' $y_0$ in eine
quadratische Funktion gegeben. Diese Parabel hat einen Sattelpunkt, der
sich bei $S(x_d/y_0)$ befindet, wobei $x_d$ der Durchschnittspreis der Stadt
representiert. Ist der Preis des Spielers also unter dem Durchschnittspreis,
werden ihm entsprechend mehr und mehr (quadratisch steigend) Kunden zugewiesen 
--- ist der Preis des Spielers unter dem Durchschnittspreis werden ihm umgekehrt
weniger Kunden als $y_0$ zugewiesen. Werden dem Spieler durch diese Funktion $y
= 0$ Kunden zugewiesen, wird der Algorithmus hier abgebrochen und der Spieler  
erh�lt keinen Vertrag mit der Stadt bzw. sein aktueller Vertrag wird gek�ndigt. 

%Bild Paralbel

Durch die Wahl einer quadratischen Funktion sind Preisabweichungen sehr
entscheidend --- anders als bei anderen Produkten und M�rkten ist in dem
Energiemarkt der Preis wichtiger als die Bekanntheit und Beliebtheit einer
Firma. 

Nun werden zwei weitere Bedingungen gepr�ft: \\
Der Spieler kann die maximale Energie festlegen, die er f�r den Vertrag
bereitstellen m�chte. Wird dieser Wert �berschritten, erh�lt er maximal so viele
Kunden, wie er beliefern kann.\\
Zuletzt muss �berpr�ft werden, ob noch genug freie Kunden in der Stadt
vorhanden sind --- es k�nnte also beispielsweise sein, dass der Spieler durch
die Preiswahl anderer Spieler aus diesem Grund keine Kunden mehr erh�lt.