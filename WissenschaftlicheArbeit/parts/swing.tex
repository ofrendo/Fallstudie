%Olli
\section{Graphische Benutzeroberfl�che}\label{olli:gui}
In dieser Sektion wird der Aufbau der graphischen Benutzeroberfl�che (GUI)
erl�utert, die mit Java Swing entwickelt wurde. Zun�chst wird das GUI durch die
Singleton Klasse \textbf{Controller} verwaltet und durch diese wird
sichergestellt, dass von dem GUI die aktuellen Informationen angezeigt werden.
Des Weiteren bietet es den GUI Klassen den Zugriff auf das Datenmodell. Die GUI
Klassen liegen im Paket \textbf{de.client.gui}. 
%In der �berwiegenden Mehrheit
%sind einzelne Komponenten durch lazy initialization

\subsection{Globale Bestandteile}
Allgemein werden die ben�tigten JPanels immer in ein Objekt der Klasse
\textbf{Frame} eingef�gt --- dieses besitzt ein \textbf{GridBagLayout}.
\begin{itemize}
  \item Ein \textbf{PanelGlobalLeft} befindet sich immer auf der linken Seite
  des Fensters, in dem die Liste von Spielern mit ihrem Status und der Chat 
  enthalten sind. Diese beiden Einzelteile werden von dem Controller
  aktualisiert, wenn der Client eine Nachricht vom Typ  
  \texttt{PLAYER\_READY\_CHANGE} bzw. \texttt{CHAT\_BROADCAST} erh�lt.
  \item Das Men� wird durch ein \textbf{PanelMenu} dargestellt, das die Buttons
  und ein \textbf{PanelMenuInformation} enth�lt. \textbf{PanelMenuInformation}
  besteht aus einem JLabel mit einem Text, der HTML Code enth�lt --- dadurch
  entsteht die Tabellenstruktur. 
\end{itemize}

Wichtig sind au�erdem die Klassen \textbf{Look}, in dem beispielsweise die
Hintergrundfarben definiert sind und \textbf{Strings}. Die Letztere stellt
mehrere Methoden bereit, die Daten f�r die Anzeige im GUI vorbereiten soll ---
darunter zum Beispiel \textbf{fD(double number)}, die Zahlen lesbarer machen
soll.

Nach dem Mockup wurden vier unterschiedliche Tabs entwickelt, die dem Spieler
unterschiedliche Einblicke in sein Unternehmen geben sollen:

\subsection{Unternehmensplanung}
Die Unternehmensplanung ist �hnlich wie die meisten anderen Tabs aus zwei
Panels aufgebaut und wird durch ein \textbf{PanelCompany} dargestellt, das von 
\\\textbf{PanelAbstractContent} erbt. 

Das linke (im Fenster mittlere) Panel ist ein \textbf{GridLayout} mit zwei
Spalten und zwei Reihen und wird lediglich von JLabels gef�llt. Die
Tabellenstruktur in den einzelnen JLabels entsteht �hnlich wie im Men� durch
die Nutzung von HTML Code. Die rechte Seite enth�lt die Nachrichten, die dem
Spieler zum Beginn der Runde angezeigt werden.

\subsection{Karte}\label{GUIKarte}
Die Karte ist das Herzst�ck der Benutzeroberfl�che --- hier kann der Spieler die
wichtigsten Entscheidungen wie das Bauen von Kraftwerken und das Schlie�en von
Vertr�gen umsetzen. Insgesamt ist der Kartentab ein \textbf{PanelMain}. Zun�chst
wird die linke (im Fenster mittlere) Seite erkl�rt.

Das \textbf{PanelMap} selbst enth�lt nur viele \textbf{HexagonButton}, dessen
Positionen von einem \textbf{HexagonLayout} verwaltet werden. Die Implementation
des HexagonLayout wurde anfangs von (\seCite{}{}{codeguru})
%\begin{center}
%http://forums.codeguru.com/showthread.php?478270-Hexagon-Buttons
%\end{center}
�bernommen. Dieses besa� am Anfang nur ein rechteckiges Layout --- es wurde
anschlie�end angepasst, damit beispielsweise die H�he und Breite der Buttons 
korrekt bei �nderung der Fenstergr��e mitskaliert werden. Des Weiteren wurde ein
hexagonf�rmiges Layout implementiert, dass mehr unserer Spielidee nachgeht: Das
GUI kann beide Kartenarten darstellen.

Jeder \textbf{HexagonButton} erh�lt bei der Initialisierung eine
\textbf{Region}, mit der er seine Hintergrundfarbe, R�nder und das Bild
bestimmten kann. Des Weiteren besitzt er eine Referenz zu einem
\textbf{PanelSubDetails}: Diese wird bei dem ersten Klick auf dem Button
initialisiert und bleibt danach erhalten. Der Inhalt davon wird durch einen
Entscheidungsbaum in \textbf{PanelDetails.setRegionContent} bestimmt.

Durch ``Lazy Initialization'' wird hier auf lange Sicht Rechenzeit gespart,
daf�r wird jedoch mehr Speicherplatz ben�tigt. Das \textbf{PanelSubDetails} wird 
dann jeweils in die rechte Seite des \textbf{PanelMap}, die durch ein 
\textbf{PanelDetails} gegeben ist, eingef�gt.

Erh�lt der Client durch eine \textbf{MAP\_UPDATE} Nachricht ist die Runde
beendet: Die Referenz zum alten \textbf{PanelMap} wird gel�scht und ein komplett
neues Objekt wird angelegt --- deswegen flackert das GUI kurz, sobald eine Runde
beendet ist.

\subsection{Finanzen}
Die finanziellen Informationen der Firma werden durch ein \textbf{PanelFinances}
abgebildet: Auf der linken Seite befindet sich ein
\textbf{PanelFinancesBalance}, das durch viele verschachtelte
\textbf{GridLayout} konstruiert wird. Daneben besteht die rechte Seite aus einem
JLabel mit HTML Code und einem \textbf{PanelCredits}.

\subsection{Markt}
Die beiden Seite des Marktes sind relativ gleich aufgebaut und sind zusammen in
einem \textbf{PanelMarket}. Sie besitzen beide ein \textbf{BoxLayout}, damit
Komponenten vertikal untereinander angezeigt werden. Die Anzeige der einzelnen
Rohstoffe, die hier gekauft und verkauft werden k�nnen, werden durch ein
\textbf{PanelMarketResource} erstellt und basieren auf der Enumeration
\textbf{de.shared.map.region.BuyableResouce}.