%Matthias 
\section{Problemstellung und Zielsetzung}
Aufgabe der vorliegenden Arbeit ist es, ein computergest�tztes
Unternehmensplanspiel zu entwickeln, wobei m�glichst viele erworbene Methoden
und Kompetenzen des Software Engineerings Anwendung finden sollen.\\
Ein Unternehmensplanspiel bezeichnet dabei "`eine modellhafte Simulation von
Unternehmensprozessen."'

Als Rahmenbedingungen wurden festgesetzt, dass eine objektorientierte Analyse
(OOA) und ein objektorientierter Entwurf (OOD) durchgef�hrt werden. Dar�ber
hinaus soll ein Prototyp der Benutzeroberfl�che (UI) erstellt werden, sowie ein
in Java implementiertes Fachkonzept. \\
Die Funktionsf�higkeit des Fachkonzepts ist durch die Verwendung von geeigneten
UnitTests sicherzustellen.

Der Projektplan sieht vor, dass die Entwicklung im Wesentlichen in mindestens
zwei Iterationen unterteilt ist, wobei nach jeder Iteration in einer
Pr�sentation die bisherigen Ergebnisse dargestellt werden sollen.


% Matthias
\section{Vorgehen}
Die ersten zwei Wochen der Planungsphase wurden darauf verwendet, einen
geeigneten Markt zu finden und die in Frage kommenden M�rkte auf ihr Potenzial
hin zu analysieren. Da diese Entscheidung ma�gebend auf alle folgenden Aufgaben
Einfluss hat und letztlich auch auf die Realisierbarkeit des
Unternehmensplanspiels, haben wir diese Zeit genutzt, um gemeinsam Brainstorm zu
betreiben, das F�r und Wider verschiedener M�rkte abzuw�gen und grundlegende
Konzepte der Spielmechanik zu evaluieren.\\
Die Kunst lag hierbei in der Konzentration auf die Ideenfindung, ohne zu sehr
ins Detail zu gehen oder sich in Themen der Analyse oder des Entwurfs zu
begeben.

Nachdem sowohl der abzubildende Markt als auch die grundlegenden Spielprinzipien
erarbeitet wurden, ging es in die Analysephase.\\
Zu Beginn der Analysephase haben wir uns �berlegt, welche Anforderungen zwingend
Bestandteil unseres Programms werden sollen, und welche optional sind und das
Spielerlebnis zwar erweitern, f�r die Funktionsf�higkeit aber nicht essentiell
sind.\\
Anschlie�end haben wir auf Basis dieser Anforderungen einen allerersten
Prototypen eines Designs f�r das Spiel entworfen, wobei es hier nur um die grobe
Darstellung der spielrelevanten Informationen und Mechanismen ging, nicht jedoch
um Einzelheiten wie Anordnungen von Buttons oder Farbschemata.

Danach haben wir die Analyse in die Pakete Server, Client-Server und
Client (Company) unterteilt. Das UI wurde abgesehen von einem ersten Prototyp
aus Zeitgr�nden und der nicht zwingenden Anforderung einer Implementierung in
der Analysephase vernachl�ssigt. Somit beschr�nkte sich unsere Analysephase im
Wesentlichen auf das Fachkonzept und die Spielmechaniken.

Bei der Analyse des Fachkonzepts stand das dabei entworfene Klassendiagramm im
Vordergrund, wobei auch ein Use Case Diagram entworfen wurde.

% Wordcount: 360